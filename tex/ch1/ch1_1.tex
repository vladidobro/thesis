\section{Maxwellovy rovnice}
\label{chap:maxwellovy-rovnice}

Pracujeme ve frekvenčním obraze, uvažujeme elektrické pole harmonické v čase s konvencí $\vec{E}(t,\vec{r})=\Re\left\lbrace\vec{E}(\omega,\vec{r}) e^{-i\omega t}\right\rbrace$ a stejně pro ostatní časově závislé veličiny.

Látky popisujeme fenomenologicky lokální\footnote{Tzv. dipólová aproximace.} lineární odezvou.
Navíc pokládáme nulovou magnetickou susceptibilitu na optických frekvencích: $\vec{M}(\omega)=0$, což kromě speciálních metamateriálů platí bez výjimky\todocite{muvac1}.
Materiály jsou za těchto podmínek plně popsány komplexními frekvenčně závislými tenzory ($3\times 3$ matice) relativní permitivity $\varepsilon_{\textrm{el}}$ a vodivosti $\sigma$
\begin{align}
    \vec{D}(\omega,\vec{r})&=\varepsilon_0 \varepsilon_{\textrm{el}}(\omega,\vec{r})\vec{E}(\omega,\vec{r}) \,, \label{eqn:materialy-D} \\
    \vec{j}(\omega,\vec{r})&=\sigma(\omega,\vec{r})\vec{E}(\omega,\vec{r}) \,, \label{eqn:materialy-J} \\
    \vec{B}(\omega,\vec{r})&=\mu_0 \vec{H}(\omega,\vec{r})  \,. \label{eqn:materialy-B}
\end{align}
Pro přehlednost budeme dále vynechávat argumenty s vyrozuměním, že vztahy platí lokálně pro všechna $\omega$ a $\vec{r}$.
Jedinou výjimkou, kdy vynechání argumentu nebude znamenat složku na frekvenci $\omega$ (jako např. $\vec{E}\equiv\vec{E}(\omega)$), bude statická magnetizace $\vec{M}\equiv\vec{M}(\omega=0)$ a statické externí pole, které značíme $\vec{H}_{\textrm{ext}}\equiv\vec{H}(\omega=0)$.

Maxwellovy rovnice v uvedené situaci mají v SI tvar
\begin{align}
    \rot \vec{E}&=i\omega\vec{B} \,, \label{eqn:Maxwell-rot-E} \\
    \rot \vec{B}&=\mu_0 \left( \sigma - i\omega \varepsilon_0 \varepsilon_{\textrm{el}} \right) \vec{E} 
        \equiv -i\omega\frac{\varepsilon}{c^2} \vec{E} \,. \label{eqn:Maxwell-rot-B}
\end{align}
Zbylé dvě divergenční Maxwellovy rovnice neuvádíme, protože pro $\omega\neq 0$ nejsou nezávislé od uvedených dvou rotačních\todocite{Visvlakna}.
V rovnicích nevystupují $\varepsilon_{\textrm{el}}$ a $\sigma$ nezávisle, ale pouze v kombinaci patrné z první rovnosti \eqref{eqn:Maxwell-rot-B}, což souvisí s tím, že rozdělení proudů na volné a vázané je pro $\omega\neq 0$ do jisté míry arbitrární.
Zavádí se proto efektivní relativní permitivita $\varepsilon$ vztahem
\begin{equation}
    \varepsilon_0 \varepsilon=\varepsilon_0 \varepsilon_{\textrm{el}}+i\sigma/\omega \,,
\end{equation}
která v sobě zahrnuje vliv všech uvažovaných proudů.
Komplexní $3\times 3$ matici $\varepsilon$ dále nazýváme zkrátka permitivitou a jedná se o jediný materiálový parametr charakterizující optické vlastnosti na dané frekvenci.
V rovnici \eqref{eqn:Maxwell-rot-B} jsme také užili rychlost světla ve vakuu $c=1/\sqrt{\mu_0 \varepsilon_0}$.

Výhodné volby rozměru rovnic dosáhneme, pokud využijeme \emph{impedanci volného prostoru} $Z_0 = \sqrt{\mu_0/\varepsilon_0}$ a vyjádříme Maxwellovy rovnice v poli $Z_0 \vec{H}=c \vec{B}$
\begin{alignat}{2}
    & \frac{1}{k_0} \left( -i \rot \right) \vec{E} &&= Z_0 \vec{H} \,, \label{eqn:rot-E}\\
    & \frac{1}{k_0} \left( -i \rot \right) Z_0 \vec{H} &&=-\varepsilon \vec{E} \,, \label{eqn:rot-ZH}
\end{alignat}
kde jsme označili $k_0 = c/\omega = 2\pi/\lambda_0$ vlnový vektor ve vakuu.
V případě rovinné vlny s prostorovou závislostí $\propto e^{i \vec{k}\cdot \vec{r}}$ platí $-i \rot = \vec{k}$, a zavádíme normovaný vlnový vektor $\vec{N}=\vec{k}/k_0$.

``Okrajové podmínky'' na rozhraní dvou materiálů, kde dochází ke skokové změně permitivity, říkají, že tečné složky $\vec{E}$ a $\vec{H}$ jsou při přechodu přes rozhraní spojité\todocite{Bornwolf}.
