\section{Polarizace}
\label{chap:polarizace}

Pro popis světla ve volném prostoru ($\varepsilon=1$) se dále omezíme na reálné $\vec{N}$.
Zvolíme osu $z$ proti směru šíření, tedy $\vec{N}=(0,0,-1)^T$.
Řešením jsou pole charakterizovaná libovolnými komplexními amplitudami $E_x$ a $E_y$, které udávají polarizaci vlny.
$\vec{H}$ je pak dáno rovnicí \eqref{eqn:rot-E}.

Výkon přenášený rovinou vlnou je dán časově středovaným Poyntingovým vektorem $\vec{S}$, pro vakuum
\begin{equation} \label{eqn:Poynting}
    \langle \vec{S}(t)\rangle_t=\langle\vec{E}(t)\times\vec{H}(t)\rangle_t=\frac{1}{2 Z_0}\vec{N} \left( E_x^*E_x+E_y^*E_y \right)
    \equiv \frac{1}{2 Z_0}\vec{N} I\,,
\end{equation}
kde definujeme normovanou intenzitu $I=E_{0x}^*E_{0x}+E_{0y}^*E_{0y}$.

Vektor $\vec{E}$ opisuje v čase obecně elipsu v rovině kolmé na směr šíření, popisné parametry definujeme podle obr. \ref{fig:polarizacni-elipsa}.
Jedná se o $\beta$ --\tododash úhel natočení hlavní poloosy polarizace, a $\chi$ -- úhel elipticity, zde dále nazývaný zkráceně elipticita.
Pro popis ekvivalentní $E_x$ a $E_y$ je třeba přidat intenzitu $I$ a časový posun $\delta$, který má význam času, ve kterém $\vec{E}(t)$ protíná hlavní poloosu s

Konvenci točivosti světla používáme následující: při podle pohledu \emph{od zdroje} obíhá pravotočivé světlo v dané rovině po směru hodinových ručiček.
Točivost se v parametrech projevuje vzájemným znaménkem $\chi$ a točivosti systému, ve kterém je $\chi$ definováno (viz další oddíl).
Viz obr. \ref{fig:polarizacni-elipsa}.

\begin{figure}[htbp]
    \centering
    \missingfigure{polarizacni elipsa}
    \caption{Polarizacni elipsa}
    \label{fig:polarizacni-elipsa}
\end{figure}

\subsection{Jonesův počet}
\label{chap:Jones}

Jonesovy vektory\todocite{jones} jsou tvořeny přímo komplexními amplitudami polí v rovině kolmé na vektor šíření
\begin{equation}
    \J=\begin{pmatrix} E_{x} \\ E_{y} \end{pmatrix} \,.
\end{equation}

Prostor Jonesových vektorů je přirozeně normovaný intenzitou \eqref{eqn:Poynting}, která je dána skalárním součinem
\begin{equation}
    \J_1^\dagger \J_2= \begin{pmatrix} J_{1x}^* & J_{1y}^* \end{pmatrix}
    \begin{pmatrix} J_{2x} \\ J_{2y} \end{pmatrix} \,, 
    \qquad I(\J)=\J^\dagger \J
\end{equation}
Dvě polarizace/Jonesovy vektory jsou ``ortogonální'' ($\J_1^\dagger \J_2=0$), pokud je celková intenzita prostý součet intenzit v obou polarizacích.

Akci každého lineární polarizačního prvku, lze popsat jako lineární transformaci Jonesova vektoru ---\tododash transformaci Jonesovou $2\times 2$ komplexní maticí $\mathcal{T}$.
Jonesovy matice lze použít pro popis polarizačního děliče: každé rameno pak má vlastní matici.

Sadě elipsometrických parametrů $\beta$, $\chi$, $I$, $\delta$ odpovídá Jonesův vektor
\begin{equation} 
\label{eqn:Jones-elipsa}
    \J=\sqrt{I} e^{i\delta} \begin{pmatrix}
        \cos \chi \cos \beta + i \sin \chi \sin \beta \\
        \cos \chi \sin \beta - i \sin \chi \cos \beta
    \end{pmatrix} \,.
\end{equation}

Zde je na místě poznámka o točivosti.
Pozdeji budeme pro usnadnění práce s téměř kolmým odrazem používat stejné souřadnice $x,y$ i pro světlo v opačném směru; vztažná soustava pak ale bude mít opačnou točivost.
V této věci se nevyhneme nějakému kompromisu, a proto přestože točivost světla je definovaná vždy vzhledem ke směru šíření, elipsometrické parametry definujeme vždy vzhledem k dané soustavě Jonesových vektorů, která může spolu se směrem šíření tvořit jak levotočivý, tak pravotočivý systém.
$\beta$ definujeme vzhledem k $J_x$, s rostoucím úhlem ve směru $+J_y$.
$\delta$ je dáno počátkem času $t=0$ a vztah mezi znaménkem $\chi$ a točivostí je dán následující poučkou: pokud $J_x J_y \vec{k}$ tvoří levotočivý systém jako na obr. \ref{fig:polarizacni-elipsa}, pak $\chi>0$ odpovídá pravotočivé polarizaci.

Při této konvenci se kruhově polarizované světlo $\J_\textrm{i}=(1, i)^T$ při kolmém dopadu odrazí se stejným Jonesovým vektorem $\J_\textrm{r}=(1, i)^T$ (a tedy stejným znaménkem $\chi$), avšak jde o opačnou točivost.


\subsection{Stokesův-Muellerův počet}
\label{chap:Stokes-Mueller}\todocite{stokesmueller}

Stokesovy parametry obecně mohou narozdíl od Jonesových vektorů popsat i nepolarizované světlo,
v této práci si však vystačíme s plně polarizovaným světlem a proto je nebudeme definovat obecně, ale pomocí Jonesových vektorů
\begin{align}
    s_0 \equiv J_x^* J_x+J_y^* J_y\equiv \J^\dagger \sigma_0 \J &= I \,,\\
    s_1 \equiv J_x^* J_x-J_y^* J_y\equiv \J^\dagger \sigma_1 \J &= I \cos 2\chi \cos 2\beta \,,  \\
    s_2 \equiv J_x^* J_y+J_y^* J_x\equiv \J^\dagger \sigma_2 \J &= I \cos 2\chi \sin 2\beta \,, \\
    s_3 \equiv i J_x^* J_y-i J_y^* J_x  \equiv \J^\dagger \sigma_3 \J &= I \sin 2\chi \,.
\end{align}
kde jsme je druhými rovnostvi vyjádřili jako střední hodnoty vhodných $2\times 2$ hermitovských matic $\sigma_{i}$. Ty jsou
\begin{align}
    \sigma_0=\begin{pmatrix} 1 & 0 \\ 0 & 1 \end{pmatrix} ,\,
    \sigma_1=\begin{pmatrix} 1 & 0 \\ 0 & -1 \end{pmatrix} ,\,
    \sigma_2=\begin{pmatrix} 0 & 1 \\ 1 & 0 \end{pmatrix} ,\,
    \sigma_3=\begin{pmatrix} 0 & i \\ -i & 0 \end{pmatrix}
\end{align}
a můžeme je rozeznat jako jednotkovou matici a přerovnané Pauliho matice.
Tvoří komplexní bázi obecných komplexních $2\times 2$ matic, s omezením na reálné koeficienty pak bázi hermitovských matic. 

Každý optický prvek lineární v Jonesových vektorech je zároveň lineární ve Stokesových parametrech
\begin{equation} 
\label{eqn:Muellerova-matice}
    s^\textrm{out}_i=\J^\dagger \mathcal{T}^\dagger \sigma_i \mathcal{T} \J\equiv\J^\dagger \left(\sum_{j=0}^{3} M_{ij} \sigma_j \right) \J
    =\sum_{j=0}^{3} M_{ij} s^\textrm{in}_j \,,
\end{equation}
kde reálná $4\times 4$ matice $\M$ je dána právě rozkladem hermitovských matic $\mathcal{T}^\dagger \sigma_i \mathcal{T}$ do báze $\sigma_j$.
Matice $\M$ charakterizující optický prvek se nazývá Muellerova matice, a sloupcový vektor $\Stks$ složený ze Stokesových parametrů se nazývá Stokesův vektor.
Složky Muellerovy matice příslušející Jonesově matici $\mathcal{T}$ je možné počítat přímo z rozkladu \eqref{eqn:Muellerova-matice} díky tzv. trace-ortogonalitě $\sigma_j$ matic: $\operatorname{Tr}\lbrace\sigma_j\sigma_i\rbrace=2\delta_{ji}$
\begin{equation} 
\label{e:mueller rozklad}
    M_{ij}=\frac{1}{2}\operatorname{Tr}\lbrace \sigma_j \mathcal{T}^\dagger \sigma_i \mathcal{T} \rbrace \,.
\end{equation}

Pro plně polarizované světlo nejsou Stokesovy parametry nezávislé, platí totiž
\begin{equation} \label{e:norma S}
s_0=\sqrt{s_1^2+s_2^2+s_3^2}
\end{equation}
a tedy nám k vyjádření polarizačního stavu stačí tři parametry $s_1, s_2, s_3$, které lze graficky zanést do třírozměrného prostoru.
Polarizace s jednotkovou intenzitou se zobrazují na tzv. Poincarého sféře jako na obr. \ref{fig:Poincareho-sfera}.
Ortogonální polarizace jsou zobrazeny na body středově souměrné podle počátku.

\begin{figure}[htbp]
    \centering
    \missingfigure
    \caption{Poincareho sfera}
    \label{fig:Poincareho-sfera}
\end{figure}

I Muellerovy matice musí splňovat určité podmínky.
Přestože libovolná myslitelná 4x4 reálná matice je dána 16 reálnými parametry, nedepolarizační (také nazývaná čistá\footnote{Ve smyslu čistého (nesmíšeného, angl. pure) stavu v kvantové mechanice.}) Muellerova matice je dána pouze 7 reálnými čisly\footnote{Jonesova matice je dána 4 komplexními čísly --- 8 reálných parametrů, ale při přechodu k Muellerově matici ztratíme informaci o celkové fázi.}. \todocite{Muellerdiff}

Dále se zaměříme na to, jakým způsobem působí obecné Muellerovy matice, a .
Muellerovy matice mají jednoduchý geometrický význam, který se graficky vyjadřuje pomocí tzv. charakteristických elipsoidů.
Charakteristický elipsoid Muellerovy matice $M$ je množina bodů v třírozměrném prostoru $(s_1, s_2, s_3)^T$, které vzniknou akcí $M$ na body ležící na Poincarého sféře.
Jinými slovy, každá Muellerova matice způsobí deformaci Poincarého sféry, výsledkem je vždy elipsoid.
V charakteristickém elipsoidu držíme i informaci o tom, na které body se transformují které body Poincarého sféry - např. fázové destičky mají za následek pouze rotaci Poincarého sféry.

\endinput
Pro popis plně polarizovaného světla se omezíme na případ čistých Muellerových matic, které vzniknou rozkladem \eqref{e:mueller rozklad} z nějaké Jonesovy matice.
Zaměříme se na dva případy: prvky reprezentované unitární Jonesovou maticí $U$ a prvky reprezentované hermitovskou pozitivně semidefinitní Jonesovou maticí $H$.

Záminku, proč zkoumat tyto dva případy $U$ a $H$, nám poskytuje věta z lineární algebry o polárním rozkladu matice\cite{pestujemealgebru}, která tvrdí, že pro každou komplexní matici $T$ existují jednoznačné rozklady $T=U H_1$ a $T=H_2 U$.\footnote{$U$ je v obou rozkladech stejné, $H_1$ a $H_2$ nemusí.}

\subsubsection*{Obecná retardační destička}

Prvek je reprezentovaný unitární Jonesovou maticí $U$.
Zachování intenzity má za důsledek $M_{00}=1$, $M_{0i}=M_{j0}=0$ pro $i,j=1,2,3$ a navíc podmatice $M{ij}$ musí zachovávat normu 3-vektoru $(S_1, S_2, S_3)$, tedy být ortonormální.
Jediné takové matice jsou 3D rotační matice, případně složené se zrcadlením.
Vzhledem k tomu, že $U$ je unitární, má dvě ortogonální vlastní polarizace $\J_1$, $\J_2$ s vlastními čísly, které jsou pouze fázové faktory. Je možné ji diagonalizovat\footnote{Ve vzorci vystupuje dyadický součin Jonesových vektorů $\J_1\J_1^\dagger$, což je ortogonální projektor na $\J_1$, ne skalární součin, který by byl psaný $\J_1^\dagger\ J_1$.}
\begin{equation}
U=e^{i\Delta_1} \J_1 \J_1^\dagger + e^{i\Delta_2} \J_2 \J_2^\dagger\,.
\end{equation}
Tyto dva vlastní módy mají po průchodu prvkem stejný polarizační stav, takže musí být i vlastními vektory Muellerovy matice, prochází jimi osa zmíněné rotace.
Úhel rotace je daný fázovým zpožděním mezi vlastními módy $\Delta_1-\Delta_2$, viz obr. \ref{f:akce muelleru} (a).

\subsubsection*{Obecný polarizátor}

Prvek je reprezentovaný pozitivně semidefinitní hermitovskou Jonesovou maticí $H$.
To znamená, že pro ně existují dvě ortogonální vlastní polarizace $J_1$, $J_2$ s reálnými nezápornými vlastními čísly.
Normalizací matice tak, že větší z vlastních čísel se rovná 1, lze psát s reálným nezáporným $\eta$
\begin{equation}
H=J_1 J_1^* + \eta J_2 J_2^* \,.
\end{equation}
Znamená to, že prvek je obecný polarizátor, který $J_1$ propustí zcela a $J_2$ propustí s amplitudovou propustností $\eta$.
Ve speciálním případě, kdy polarizátor propouští lineární polarizaci v ose $x$: $J_1=(1,0)^\T$ a $J_1=(0,1)^\T$, je Jonesova a Muellerova matice
\begin{equation}
H=\begin{pmatrix}
1 & 0 \\ 0 & \eta
\end{pmatrix} \,, \qquad
M_H=\begin{pmatrix}
\frac{1+\eta^2}{2} & \frac{1-\eta^2}{2} & 0 & 0 \\ \frac{1-\eta^2}{2} & \frac{1+\eta^2}{2} & 0 & 0 \\
0 & 0 & \eta & 0 \\ 0 & 0 & 0 & \eta
\end{pmatrix} \,.
\end{equation}

Pro výpočet charakteristického elipsoidu dosadíme $S_0^{\textrm{in}}=1$ a dostaneme
\begin{align}
    S_1^{\textrm{out}}&=\frac{1+\eta^2}{2} S_1^{\textrm{in}}+\frac{1-\eta^2}{2} \\
    S_2^{\textrm{out}}&=\eta S_2^{\textrm{in}} \\
    S_3^{\textrm{out}}&=\eta S_3^{\textrm{in}}
\end{align}
Jedná se tedy o kontrakci v rovině $S_2S_3$ faktorem $\eta$, ve směru $S_1$ faktorem $(1+\eta^2)/2$ a zároveň posunutím o $(1-\eta^2)/2$.
Nebo ekvivalentně kontrakcí stejným faktorem se středem v $S_3=1$. Viz obr. \ref{f:akce muelleru} (b).

\begin{figure}\centering
\includegraphics[width=\linewidth]{./img/t3.png}
\caption{Grafické zobrazení akce (a) obecné retardační destičky a (b) obecného polarizátoru.}\label{f:akce muelleru}
\end{figure}

Shrneme-li uvedené poznatky, akce libovolného nedepolarizačního optického prvku je ekvivalentní postupnému působení obecného polarizátoru (zploštění a posunutí ve směru vlastního vektoru $H$ jako na obr. \ref{f:akce muelleru} (b)) a obecné fázové retardační destičky (rotace podle směru vlastního vektoru $U$ jako na obr. \ref{f:akce muelleru} (a)), případně v opačném pořadí.
