\section{Magnetická anizotropie}
\label{chap:magneticka-anizotropie}

Magnetizace $\vec{M}$ materiálu není veličina, která by se dala v experimentu přímo ovládat.
V experimentu můžeme aplikovat vnější pole $\vHext$ a materiál si sám najde rovnovážnou polohu $\vec{M}$.
Na druhou stranu magnetooptické vlastnosti, jak je patrné z předchozího oddílu, závisí od $\vec{M}$.
Tento oddíl se věnuje vztahu mezi $\vec{M}$ a $\vHext$.

Pro ilustraci budeme uvažovat systém s homogenní magnetizací.
Z vnitřní energie vyčleníme energii samotného vnějšího magnetického pole, které by bylo přítomno i při absenci systému, a zahrneme pouze magnetickou energii spojenou se systémem tak, jak je popsáno v dodatku učebnice \cite{callenThermodynamicsIntroductionThermostatistics1985}.
Vnější pole $\vHext$ tedy uvažujeme jako čistě pole tvořené externími zdroji a nezahrnuje demagnetizační pole.
Magnetický příspěvek hustoty vnitřní energie pak má tvar\cite{callenThermodynamicsIntroductionThermostatistics1985} 
\begin{equation}
        \textrm{d} U_\textrm{mag} = \mu_0 \vHext \cdot \textrm{d}\vec{M}\,.
\end{equation}

Pokud je udržovaný na teplotě $T$, udává jeho termodynamické vlastnosti hustota volné energie $F(T,\,\vec{M})$\cite{callenThermodynamicsIntroductionThermostatistics1985}.
Aby mohl být systém v rovnováze při konkrétním $\vec{M}$, musí externí pole jakožto přidružený intenzivní parametr splňovat
\begin{equation} 
\label{eqn:Hext=gradF}
    \mu_0\Hext(\vec{M}) = \nabla_{\vec{M}} F(\vec{M}) \,.
\end{equation}

Radši bychom ale znali rovnovážné $\vec{M}$ v situaci, kdy je systém obklopen magnetickým polem $\vHext$ tvořeným např. cívkami elektromagnetu.
Mezi magnetem a studovaným systémem dochází k výměně energie prostřednictvím magnetického pole, systém je v kontaktu s "magnetickým rezervoárem" a v rovnováze proto dochází k minimalizaci \emph{celkové} volné energie.
V souladu s teorií termodynamických potenciálů tedy přejdeme k Legendrově transformaci v $\vHext$ -- hustotě ``magnetické entalpie''\footnote{Někdy označované jako magnetický Gibbsův potenciál.} systému\cite{castellanoThermodynamicPotentialsSimple2003}
\begin{equation} 
\label{eqn:magneticka-entalpie}
    \Omega(T,\vHext)= -\mu_0\vHext\cdot\vec{M}(\vHext)+F\left(\vec{M}(\vHext)\right) \,.
\end{equation}
Princip minima termodynamických potenciálů nám říká, že pro libovolné pevné $\vHext$ bude $\vec{M}$ nabývat takové hodnoty, která minimalizuje $\Omega$.
V obecné situaci, kdy magnetizace není homogenní a jednotlivá místa systému spolu interagují, může volná energie být obecným nelokálním funkcionálem prostorového rozložení magnetizace.
Hustota $F(\vec{M})$ se proto nazývá \emph{funkcionál volné energie}.

Široce používaný model feromagnetů v jedno-doménovém stavu je tzv. Stonerův-Wohlfarthův model\cite{stonerMechanismMagneticHysteresis1991}.
Předpokládá, že funkcionál $F(\vec{M})$ má význam pouze lokální hustoty a $\vec{M}$ je tedy dána minimalizací \eqref{eqn:magneticka-entalpie}.
Ve formě, v jaké SW model budeme používat, zahrnujeme do volné energie 4 příspěvky\cite{reichlovaUltrarychlaLaserovaSpektroskopie2010,jandaDynamikaSpinovePolarizace2012,kucharikStudiumSpinovePolarizace2015}
\begin{equation}
    F=F^\textrm{exchange} + F^\textrm{magnetocrystalline} + F^\textrm{shape} + F^\textrm{strain} \,.
\end{equation}
První člen, způsobený výměnnou interakcí, má na svědomí feromagnetismus; závisí na celkové velikosti magnetizace $|\vec{M}|$ a má ostré minimum pro saturovanou magnetizaci $|\vec{M}|=M_S$ -- všechny mikroskopické magnetické momenty jsou orientované stejným směrem.
Magnetokrystalická anizotropie $F^\textrm{magnetocrystalline}$ popisuje interakci s krystalickou mřížkou, tvarová anizotropie $F^\textrm{shape}$ popisuje vliv tvaru vzorku a strainová anizotropie $F^\textrm{strain}$ popisuje anizotropie způsobené mechanickým napětím (např. když je vzorek nanesen na substrátu s odlišnou mřížkovou konstantou).

Dále se omezíme na situaci relevantní pro tuto práci.
Vzorek je feromagnetický a vlivem výměnné interakce je magnetizace vždy saturovaná.
Vzorek je kubický tenký film s krystalografickými směry [100], [010] a [001] shodnými s kladnými poloosami $x$, $y$, $z$.
Tvarová anizotropie způsobí vymizení out-of-plane magnetizace $M_z=0$
\begin{equation}
\label{eqn:magnetizace-v-rovine}
    \vec{M}=\begin{pmatrix} M_x \\ M_y \\ M_z \end{pmatrix}
    = M_S \begin{pmatrix} \cos \phim \\ \sin \phim \\ 0 \end{pmatrix} .
\end{equation}

Magnetokrystalická anizotropie se rozvine do mocninné řady v $\vec{M}$ a ponechá se pouze nejnižší člen respektující kubickou symetrii, s uvážením \eqref{eqn:magnetizace-v-rovine}
\begin{equation}
    \frac{F^\textrm{magnetocrystalline}}{M_S}=k_4 \sin^2 \phim \cos^2 \phim \,,
\end{equation}
čímž jsme definovali kubickou anizotropní konstantu $k_4$.
Pro $k_4>0$ má minima -- snadné osy -- ve směrech [100] a [010] (tj. $\phim=\SI{0}{\degree}$, \SI{90}{\degree}), pro $k_4<0$ jsou to [110] a [1-10] (tj. $\phim=\SI{45}{\degree}$, \SI{135}{\degree}).

Navíc povolíme uniaxiální strainovou anizotropii.
Také ji rozvineme do řady a podpořeni tím, že mechanické napětí v rovině má principiálně uniaxiální charakter, ponecháme pouze první člen a opět vydělíme $M_S$ pro definici uniaxiální anizotropní konstanty $k_u$ a směru $\phiu$.
\begin{equation}
    \frac{F^\textrm{strain}}{M_S}=k_u \sin^2\left( \phim-\phiu  \right) \,.
\end{equation}
$\phiu$ je takto definováno vzhledem ke krystalografickému směru [100].
Je dostačující omezit se na $k_u\geq 0$, snadné směry jsou pak ve $\phim=\phiu, \phiu+\SI{180}{\degree}$.
Obě hodnoty $\phiu$ a $\phiu+\SI{180}{\degree}$ popisují stejné $F^\textrm{strain}$, takže pokud v konkrétním případě nemáme důvod konat jinak (např. z důvodu spojitosti), omezujeme se na $\phiu \in [\SI{0}{\degree}, \SI{180}{\degree}]$.

Kanonický tvar funkcionálu volné energie tenkého kubického filmu v rovině $xy$ orientovaného $[100]=x$ tedy píšeme
\begin{equation}
\label{eqn:SW-funkcional}
    \frac{F(\phim)}{M_S}=k_4 \sin^2 \phim \cos^2 \phim + k_u \sin^2\left( \phim-\phiu  \right) \,.
\end{equation}

Pro $\vHext$ v rovině $xy$
\begin{equation}
    \vHext =\Hext \begin{pmatrix} \cos \phih \\ \sin \phih \\ 0 \end{pmatrix}
\end{equation}
je $\phim(\phih)$ dáno minimalizací hustoty magnetické entalpie (vydělené konstantním $M_S$)
\begin{equation}
    \label{eqn:magneticka-entalpie-v-rovine}
    \frac{\Omega}{M_S}=-\mu_0 \Hext \cos \left(\phim-\phih \right) + k_4 \sin^2 \phim \cos^2 \phim + k_u \sin^2\left( \phim-\phiu  \right) \,.
\end{equation}
Dělení $M_S$ zavádíme, aby anizotropní konstanty $k_4$ a $k_u$ měly dimenzi magnetického pole a byly přímo porovnatelné s experimentálně ovladatelným $\mu_0\Hext$, bez nutnosti znalosti $M_S$.

Pro praktické účely je výhodné vyjádřit \eqref{eqn:SW-funkcional} ekvivalentním způsobem pro vzorek obecně natočený v rovině $xy$ o úhel $\gamma$, tzn. [100] je ve směru vektoru $(\cos\gamma, \sin\gamma, 0)$.
Pak až na bezvýznamnou aditivní konstantu
\begin{subequations}
\label{eqn:funkcional-otoceny}
\begin{align}
    \frac{F(\phim)}{M_S}&=-\frac{k_4}{8} \cos 4(\phim-\gamma)-\frac{k_u}{2} \cos 2(\phim-\gamma-\phiu) \\
                        &=\tilde{k}_4 \frac{-e^{-i4\phim}}{16} + \tilde{k}_u \frac{-e^{-i2\phim}}{4} + \textrm{c. c.}\,,
\end{align}
\end{subequations}
kde c. c. značí komplexně sdružený předešlý výraz a neobvyklá normalizace je volena tak, aby
\begin{equation}
\label{eqn:tilda-k}
    \tilde{k}_4 = k_4 e^{i4\gamma} \,, \quad \tilde{k}_u = k_u e^{i2(\gamma+\phiu)} \,.
\end{equation}

Pro in-plane magnetickou anizotropii se zavádí tzv. \emph{torque}\footnote{Česky točivý moment.} jako derivace volné energie $L=\allowbreak-\textrm{d}F/\textrm{d}\phim$.
Podmínka minima $\Omega$ má pak tvar
\begin{equation}
    \label{eqn:torque}
    L(\phim) = \mu_0\Hext M_S \sin (\phim-\phih) \,.
\end{equation}

Existence volné energie klade netriviální požadavky na průběh $\vec{M}(\vHext)$ -- podmínku integrability.
Pro relevantní situaci saturované in-plane magnetizace a rotujícího vnějšího pole konstantní velikosti má tvar
\begin{equation}
\label{eqn:podminka-integrability}
    \mu_0\Hext M_S \int_{0}^{2\pi}  \frac{\text{d}\phim}{\text{d}\phih} \sin\left(\phim-\phih\right) \text{d}\phih=0 \,,
\end{equation}
za podmínky, že $\phim$ je spojitou funkcí $\phih$ -- nedochází k přeskokům magnetizace.

Důsledkem je např. intuitivní fakt, že z myslitelných průběhů $\phim(\phih)=\phih+c$ je jediný možný ten, pro který $c=0$;
není možné, aby magnetizace konzistentně ``předbíhala'' nebo se ``opožďovala'' za přiloženým polem.
Pokud z experimentu dokážeme určit pouze $\text{d}\phim/\text{d}\phih$, podmínka \eqref{eqn:podminka-integrability} nám dovoluje určit integrační konstantu.
Podrobnosti jsou uvedeny v dodatku \ref{app:magneticka-anizotropie}.
