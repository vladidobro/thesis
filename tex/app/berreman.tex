\section{Průchod a odraz při poruchách permitivity}
\label{app:berreman}

Navazujeme zde na oddíl \ref{chap:optika-v-multivrstvach} a ukážeme, jak spočítat Jonesovy transmisní a reflexní matice při malých změnách tenzoru permitivity, tj. budeme se snažit vyjádřit derivace $\delta\Tjs$ a $\delta\Rjs$ podle prvků Berremanovy matice $\delta\Delta$.
Podobný výpočet byl do jisté míry naznačen v článku \cite{bertrandGeneralAnalyticalTreatment2001}.

Omezíme se na reálný normovaný příčný vlnový vektor $N$ pro popis laserových svazků.

\subsection*{Porucha reflexní a transmisní matice}

Označíme celkovou přenosovou matici \eqref{eqn:prenosova-matice-M} v bázi módových amplitud vstupního a výstupního prostředí
\begin{equation}
    \Nfr = \mathfrak{D}_{(L)}^{-1} \Mfr  \mathfrak{D}_{(R)} \,.
\end{equation}
Také zavedeme blokové značení $2\times2$ bloků $4\times4$ matic
\begin{equation}
    \Nfr = \begin{pmatrix} \Nfr^\nwarrow & \Nfr^\nearrow \\
    \Nfr^\swarrow & \Nfr^\searrow \end{pmatrix} .
\end{equation}
S tímto značením jsou Jonesovy matice průchodu a odrazu po vyřešení rovnice \eqref{eqn:rovnice-odraz-pruchod} dány
\begin{equation}
    \Tjs = \left(\Nfr^\nwarrow\right)^{-1} \,, \quad \Rjs=\Nfr^\swarrow \mathcal{T} \,.
\end{equation}
Poruchy matice $\delta\Nfr$ se v Jonesových maticích projeví\footnote{Derivace inverzní matice $(A^{-1})'=-A^{-1}A'A^{-1}$.}
\begin{equation}
    \label{eqn:app-derivace-RT}
    \delta\Tjs = -\Tjs \left(\delta\Nfr^\nwarrow \right) \Tjs \,, \quad \delta\Rjs=\left(\delta\Nfr^\swarrow\right) \Tjs + \Nfr^\swarrow (\delta\Tjs)
\end{equation}

\subsection*{Porucha přenosové matice}
\subsubsection*{Sendvičová struktura}

Zde se zaměříme na výpočet $\delta\Nfr$ v případě, kdy vrstva s porušenou permitivitou není přímo výstupní prostředí.
Potom $\DfrL$ i $\DfrR$ zůstávají neměnné a $\delta\Nfr = \DfrL^{-1}\delta\Mfr\DfrR$.
Celková přenosová matice $\Mfr$ je dána součinem tří dílčích přenosových matic: nadvrstev $\Mfr_i$, zkoumané (magnetické) vrstvy s porušenou permitivitou $\Mfr_m$ a podvrstev $\Mfr_o$, takže $\Mfr = \Mfr_i \Mfr_m \Mfr_o$.
Na poruchu pak reaguje pouze $\Mfr_m$, tzn. $\delta\Mfr = \Mfr_i (\delta\Mfr_m) \Mfr_o$.

Přenosová matice je dána řešením \eqref{eqn:Berreman-master}, maticovou exponenciálou
\begin{equation}
    \Mfr_m + \delta \Mfr_m = e^{ik_0d(\Delta+\delta\Delta)} = e^{ik_0d\Delta} + \delta \Mfr_m
\end{equation}
kde $d$ je tloušťka vrstvy.

První derivaci maticové exponenciály lze spočítat podle vzorce\cite{najfeldDerivativesMatrixExponential1995a}
\begin{equation}
    \label{eqn:app-derivace}
    \delta\Mfr_m = \delta(e^{ik_0d\Delta}) = ik_0d\int_0^1 e^{ik_0d\Delta(1-\tau)} (\delta\Delta) e^{ik_0d\Delta \tau} \textrm{d}\tau \,.
\end{equation}
Není tedy třeba počítat exponenciálu porušených $\Delta+\delta\Delta$, ale pouze neporušeného $\Delta$.
Pokud dokážeme neporušenou matici $\Delta$ diagonalizovat dynamickou maticí $\Dfr$, má vzorec \eqref{eqn:app-derivace} jednodušší tvar
\begin{equation}
    \label{eqn:app-lepsi-derivace}
    \delta\Mfr_m = ik_0d \Dfr \int_0^1 e^{ik_0d (\Dfr^{-1}\Delta\Dfr)(1-\tau)} (\Dfr^{-1}\delta\Delta\Dfr) e^{ik_0d(\Dfr^{-1}\Delta\Dfr)\tau} \textrm{d}\tau \Dfr^{-1}\,,
\end{equation}
kde se už vyskytuje exponenciála pouze diagonálních matic $\Dfr^{-1}\Delta\Dfr$, která se spočítá triviálně.
Pro přehlednost explicitně spočítáme derivaci v případě, kdy neporušená permitivita je izotropní, tj. v $2\times2$ blokové notaci
\begin{equation}
    \Dfr^{-1}\Delta\Dfr = \begin{pmatrix} -n\cos\alpha_t & 0 \\ 0 & n\cos\alpha_t \end{pmatrix} ,
\end{equation}
kde $n$ je index lomu a $\alpha_t$ úhel lomu\footnote{Pro absorbující prostředí je komplexní, nicméně stále platí $\cos\alpha_t\equiv\sqrt{1-N^TN/n^2}$.}.
Pak
\begin{equation}
    \label{eqn:app-zmena-exp}
    \delta \exp \left[ik_0d\Delta+ \begin{pmatrix} 
\mathfrak{A}^\nwarrow & \mathfrak{A}^\nearrow \\
\mathfrak{A}^\swarrow & \mathfrak{A}^\searrow
\end{pmatrix} \right] = ik_0d \, \Dfr
\begin{pmatrix} 
    e^{-ix}\mathfrak{B}^\nwarrow & \frac{\sin x}{x}\mathfrak{B}^\nearrow \\
    \frac{\sin x}{x}\mathfrak{B}^\swarrow & e^{ix}\mathfrak{B}^\searrow
\end{pmatrix} \Dfr^{-1} \,,
\end{equation}
kde jsme označili $x = k_0 d n \cos\alpha_t$ a matice $\mathfrak{B} = (b_{ij})$ vznikne z poruchy $\mathfrak{A} = (a_{ij})$ transformací $\mathfrak{B} = \Dfr^{-1} \mathfrak{A} \Dfr$ jako \eqref{eqn:app-lepsi-derivace}.
Každý z $2\times2$ bloků je násoben svým vlastním činitelem, takže výsledek nelze jednoduše zapisovat pomocí notace klasického maticového násobení.

\subsubsection*{Odraz od bulku}

Druhý případ, kdy magnetická vrstva je přímo výstupní prostředí (ale mezi ní a vstupním prostředím můžou být další vrstvy), je složitější.
Pokud se zajímáme i o prošlé světlo, narážíme na problém, který je jinak všudypřítomný v Yehově formalismu, totiž dynamická matice je v okolí degenerací (izotropní vrstva či šíření podél optické osy) singulární.
Ta v anizotropním prostředí nutně definuje bázi Jonesových vektorů\footnote{Protože oba svazky jsou oddělené.}, což má za následek neexistenci derivace $\delta\mathcal{T}$.
Pokud bychom přece jen chtěli popsat změny prošlého světla, museli bychom ho místo módovými amplitudami (Jonesovými vektory) popisovat pomocí složek polí $\vec{E}$, $\vec{H}$, které singularitami netrpí.

Častější situace, na kterou se zde zaměříme, je taková, že se zajímáme pouze o odražené světlo (odraz na anizotropním bulku).
Zmíněná singularita $\Tjs$ se pak objevuje pouze v mezivýpočtech a je možné jí obejít.

Z činitelů přenosové matice jsou nyní neměné $\DfrL$ a $\Mfr$, takže po derivaci $\delta\Nfr=\DfrL^{-1}\Mfr(\delta\DfrR)$.
Rovnici \eqref{eqn:rovnice-odraz-pruchod} přepíšeme do vhodnějšího tvaru
\begin{equation}
    \Nfr^{-1} \begin{pmatrix} \J^i \\ \J^r \end{pmatrix}
    = \begin{pmatrix} \J^t \\ 0 \end{pmatrix} ,
\end{equation}
řešením pro $\J^r$ je pak
\begin{equation}
    \label{eqn:app-projekce}
    \Kfr^\swarrow \J^i + \Kfr^\searrow \J^r = 0
\end{equation}
s označením $\Kfr=\Nfr^{-1} = \DfrR^{-1}\Mfr^{-1}\DfrL$.
Výsledná reflexní matice je pak 
\begin{equation}
    \label{eqn:app-refl-bulk}
\Rjs=-(\Kfr^\searrow)^{-1}\Kfr^\swarrow \,.
\end{equation}
Reflexní matice má derivaci, ale nelze jí počítat tímto způsobem, protože derivace $\DfrR^{-1}$ neexistuje v okolí degenerací.

Kdybychom se pídili po původu této singularity, došli bychom k následujícímu závěru.
Význam rovnice \eqref{eqn:app-projekce} je takový, že nutí ty kombinace složek příčných polí $\vec{G}$, které odpovídají módům přicházejícím ke struktuře z výstupního prostředí, aby byly nulové (z druhé strany nikdo nesvítí).
Projekce $\vec{G}$ na dvou-dimenzionální podprostor\footnote{Projekce je šikmá, ve skutečnosti se jedná o doplněk kolmé projekce na podprostor dvou propuštěných módů.} těchto dvou módů musí být nulová.
Singularita vzniká z toho důvodu, že v \eqref{eqn:app-projekce} je díky přítomnosti $\DfrR$ vždy volena báze vlastních vektorů příslušné porušené matice $\Delta+\delta\Delta$.
Řešením je přejít od dynamických matic k projektorům na spektrální podprostory, nebo ekvivalentně používat takovou sadu\footnote{Sadou bází rozumíme bázi podprostoru pro každé dané $\delta\Delta$.} bází, která singulární není.
Taková sada bází pak není tvořena přímo vlastními módy, ale tvoří stejný podprostor, takže je po zámeně s původní bází v \eqref{eqn:app-projekce} ekvivalentní.
Výsledkem je, že v \eqref{eqn:app-projekce} nahradíme $\Kfr$ takovým $\tilde{\Kfr}$, se kterým je rovnice ekvivalentní, ale navíc $\tilde{\Kfr}$ lze derivovat.

Nyní ukážeme, jak takové $\tilde{\Kfr}=(\DfrRt)^{-1}\Mfr^{-1}\DfrL$ spočítat a derivovat.
Potom je porušená reflexní matice (derivací \eqref{eqn:app-refl-bulk})
\begin{equation}
    \label{eqn:app-dR-bulk}
    \delta\Rjs = -(\Kfr^\searrow)^{-1} (\delta\Kfr^\swarrow) + (\Kfr^\searrow)^{-1} (\delta\Kfr^\searrow)(\Kfr^\searrow)^{-1}\Kfr^\swarrow
\end{equation}
a $\tilde{\Kfr}$
\begin{equation}
    \delta\tilde{\Kfr} = \left[-(\DfrRt)^{-1} (\delta\DfrRt) (\DfrRt)^{-1} \right]\Mfr^{-1}\DfrL \,.
\end{equation}
K singularitám dochází pouze, pokud jsou dva nežádoucí módy degenerované, tj. neporušená $\Delta$ má v bázi neporušených módů tvar
\begin{equation}
    \Dfr^{-1}\Delta\Dfr = \begin{pmatrix} 
        -a_1&0&0&0\\0&-a_2&0&0\\
        0&0&b&0\\0&0&0&b
    \end{pmatrix},
\end{equation}
kde jsme už v neporušené $\Delta$ dovolili nedegenerované propuštěné módy ($-a_1$ a $-a_2$) a znaménkem jsme naznačili, které módy postupují kterým směrem.



Pokud se zajímáme o poruchy v okolí již anizotropní permitivity, je možné derivace $\DfrR$ a tedy i $\Nfr$, $\Tjs$ a $\Rjs$ počítat standardními metodami poruchového počtu pro výpočet derivací vlastních vektorů\footnote{S tím rozdílem, že $\Delta$ není hermitovská.}.

\subsection*{Diskuze}

Uvedený výpočet je sice nepřehledný a vyžaduje dávku účetnického talentu, na druhou stranu je ale přímočarý a dovoluje počítat změny stočení a elipticity za pomoci pouze lineární algebry a znalosti přenosové a dynamické matice neporušených vrstev.
V nejčastější situaci, kdy neporušená struktura je tvořena pouze izotropními vrstvami, je jejich výpočet navíc velice jednoduchý\cite{vigoureuxPolynomialFormulationReflection1991}, redukovaný na výpočet Abelových $2\times2$ přenosových matic v izotropních vrstvách.

Podobný výpočet nebyl dosud proveden.
V článku \cite{hamrleVicinalInterfaceSensitive2003} byl spočítán speciální případ ultra-tenké vrstvy, která odpovídá v rovnici \eqref{eqn:app-zmena-exp} limitě $x\to0$. 
Naše rozšíření je platné i pro \emph{středně}-tenké vrstvy, pro které nemusí platit $nd\ll\lambda$ jako pro ultra-tenké vrstvy, ale slabší podmínka $(\Delta n)d\ll\lambda$, kde jsme označili míru zanesené anizotropie rozštěpením indexu lomu $\Delta n$.

V článku \cite{postavaAnisotropyQuadraticMagnetooptic2002} byl spočítán odraz přímo od bulku (s izotropní neporušenou permitivitou) se stejným výsledkem \eqref{eqn:QMOKE-vzorec}, takže standardní postup je předpokládat platnost \eqref{eqn:QMOKE-vzorec} i mimo tyto dva limitující případy\cite{hamrleHugeQuadraticMagnetooptical2007,kuschelVectorialMagnetometryUsing2011}.
Náš výpočet zaplňuje tuto důležitou mezeru a dovoluje počítat, jak MO koeficienty závisí na tloušťce vrstvy.

Další výhodou našeho výpočtu je také to, že není nutné se omezovat na izotropní neporušené permitivity.
Výsledky jsou platné i pokud zkoumáme odchylky permitivity již anizotropních struktur.

Samozřejmě lze úlohu vždy řešit numericky, poskytujeme ale rychlý způsob určování analytických výrazů pro ``extrakční faktory'' (např. vážící konstanty $A_{s/p}$, $B_{s/p}$ z \eqref{eqn:QMOKE-vzorec}) v mnohem širší třídě situací, než bylo doposud možné.

\subsubsection*{Obecné výsledky}

Pokud je multivrstva tvořena pouze izotropními vrstvami, mají všechny dynamické i přenosové matice speciální tvar.
Všechny jejich $2\times2$ bloky musí být diagonální (jako \eqref{eqn:D-sp}) a to nám umožňuje učinit určité závěry ohledně tvaru $\delta\Rjs$ a $\delta\Tjs$ bez konkrétních výpočtů.
V obou případech sendvičové struktury i odrazu od bulku (v případě izotropní struktury $a_1=a_2$) je posloupnost kroků vedoucí k výpočtu $\delta\Rjs$ a $\delta\Rjs$ tvořena pouze několika druhy operací.
Buď může být každý blok matice vynásoben svým vlastním faktorem (jako \eqref{eqn:app-zmena-exp}), nebo může být matice z obou stran násobena některou z přenosových nebo dynamických matic, a nebo z ní může být extrahován jeden z $2\times2$ bloků (jako např. \eqref{eqn:app-dR-bulk}).
Výsledkem libovolné posloupnosti těchto tří operací může být pouze lineární kombinace bloků $\delta\Delta$ násobené zleva a zprava nějakými diagonálními maticemi.
To znamená, že
\begin{equation}
    \label{eqn:app-ktere-cleny}
    \begin{split}
        \delta\Tjs \,, \delta\Rjs = \begin{pmatrix} c_{11}&0\\0&c_{12} \end{pmatrix} \delta\Delta^\nwarrow \begin{pmatrix} c_{13}&0\\0&c_{14} \end{pmatrix} 
 +\begin{pmatrix} c_{21}&0\\0&c_{22} \end{pmatrix} \delta\Delta^\nearrow \begin{pmatrix} c_{23}&0\\0&c_{24} \end{pmatrix} + \ldots
\end{split}
\end{equation}
s nějakými konstantami $c_{ij}$.
Po uvážení tvaru Berremanovy matice \eqref{eqn:Berreman-master} nám to poskytuje užitečnou pomůcku jak určovat, v kterých prvcích Jonesových matic se mohou uplatňovat které prvky permitivity (např. $\varepsilon_{12}$ se může projevit pouze v $\Rjs_{sp}$ a nikoli $\Rjs_{ss}$, $\Rjs_{pp}$, $\Rjs_{ps}$).
Závěrem je, že i v našem případě středně-tenké vrstvy platí obecnější tvar vzorce \eqref{eqn:QMOKE-vzorec}, který pochází z \cite{hamrleVicinalInterfaceSensitive2003}
\begin{align}
    r_{ps} &\propto A_s \left( \varepsilon_{21} - \frac{\varepsilon_{23}\varepsilon_{31}}{\varepsilon_{33}}\right) + B_s \varepsilon_{31} \\
r_{sp} &\propto A_p \left( \varepsilon_{12} - \frac{\varepsilon_{32}\varepsilon_{13}}{\varepsilon_{33}}\right) + B_s \varepsilon_{13} \,.
\end{align}


\subsubsection*{Výsledky v kolmém dopadu}

Pokud jsou všechny vrstvy izotropní a světlo dopadá kolmo ($N=0$), je výpočet velice jednoduchý.
Jediný nenulový blok poruchy Berremanovy matice \eqref{eqn:Berreman-master} je levý spodní
\begin{equation}
    \delta\Delta = \begin{pmatrix} 0 & 0\\
    \varepsilon^\perp - \frac{\varepsilon^\vert\varepsilon^-}{\varepsilon_{33}+n^2}& 0\end{pmatrix},
\end{equation}
kde $\varepsilon$ nyní značí pouze odchylku od izotropního indexu lomu $n$, a dynamické matice tvoří pouze bloky úměrné jednotkové matici (viz \eqref{eqn:D-sp})
\begin{equation}
    \Dfr = \begin{pmatrix} 1&1\\-n&n \end{pmatrix} .
\end{equation}

Ve vzorci \eqref{eqn:app-ktere-cleny} jsou pak všechny z $c$ matic diagonální a jediný člen je úměrný $\delta\Delta^\swarrow$, takže můžeme psát
\begin{equation}
    \Rjs + \delta\Rjs = R \left[ \begin{pmatrix} 1&0\\0&1 \end{pmatrix} + P_0 \left( \varepsilon^\perp - \frac{\varepsilon^\vert\varepsilon^-}{\varepsilon_{33}+n^2} \right) \right]
\end{equation}
a stejně pro transmisní matici.


Pro kubický vzorek orientovaný hlavními krystalografickými směry v osách $x$, $y$, $z$ můžeme dosadit $\varepsilon(\vec{M})$ z \eqref{eqn:permitivita-kub-K} a \eqref{eqn:permitivita-kub-G-xy} a dostat relevantní část Berremanovy matice
\begin{align}
    \varepsilon^\perp - \frac{\varepsilon^\vert\varepsilon^-}{\varepsilon_{33}+n^2} =& \frac{1}{2}\left( \frac{G_s}{2}-\frac{K^2}{n^2} \right) (\sigma_1\cos2\phim + \sigma2\sin2\phim) \\
                                                                                &+ \frac{1}{2}\frac{\Delta G}{2}(\sigma_1\cos2\phim - \sigma_2 \sin2\phim) \,.
\end{align}
Z toho již plyne model \eqref{eqn:PMLD-kubicke}.
