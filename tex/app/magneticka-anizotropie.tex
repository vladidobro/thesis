\section{Dodatky k magnetické anizotropii}
\label{app:magneticka-anizotropie}

\todo{revize}

\subsection*{Podmínky integrability}

Zvolíme-li uzavřenou křivku v $\M$ prostoru, a budeme podél ní integrovat \eqref{e:Hext=gradF}, dostaneme
\begin{equation} \label{e:mag integral 1}
    \oint \mu_0\vHext(\vec{M}) \cdot \textrm{d}\vec{M} = \oint \nabla_{\vec{M}} F \cdot \text{d}\vec{M} = F(\vec{M}_\textrm{start})-F(\vec{M}_\textrm{end})=0 \,.
\end{equation}
Přechodem k experimentálně ovladatelným souřadnicím $\vHext$
\begin{equation} \label{e:substituce M}
\textrm{d}\M=\begin{pmatrix}\textrm{d}M_x \\ \textrm{d}M_y \\ \textrm{d}M_z \end{pmatrix}
=\begin{pmatrix}
\frac{\partial M_x}{\partial H_x} & \frac{\partial M_x}{\partial H_y} & \frac{\partial M_x}{\partial H_z} \\
\frac{\partial M_y}{\partial H_x} & \frac{\partial M_y}{\partial H_y} & \frac{\partial M_y}{\partial H_z} \\
\frac{\partial M_z}{\partial H_x} & \frac{\partial M_z}{\partial H_y} & \frac{\partial M_z}{\partial H_z} \\
\end{pmatrix}
\begin{pmatrix}\textrm{d}H_x \\ \textrm{d}H_y \\ \textrm{d}H_z \end{pmatrix}
\equiv\left( \frac{\textrm{d}\vec{M}}{\textrm{d}\vHext} \right) \textrm{d}\vHext
\end{equation}
dostáváme z \eqref{e:mag integral 1} podmínku na průběh $\vec{M}(\vHext)$ po uzavřené křivce $\vHext$ 
\begin{equation}
    \oint \mu_0 \vHext \cdot \left( \frac{\textrm{d}\vec{M}}{\textrm{d}\vHext} \right) \textrm{d}\vHext = 0 \,,
\end{equation}
která platí, pokud jsme mohli provést substituci \eqref{e:substituce M}, tj. pokud je na ní $\vec{M}(\vHext)$ spojité, nedochází k přeskokům.
Striktně vzato bychom ji mohli použít i v případě, kdy dochází k vratným\footnote{Ve smyslu vratného termodynamického procesu.} přeskokům, tj. výchozí i koncové $\vec{M}$ mají stejnou volnou energii; nedochází k hysterezi.

Přímým dosazením pro případ, kdy se $\vHext$ otočí v rovině $xy$ o \SI{360}{\degree} s konstantní amplitudou a saturovanou in-plane $\vec{M}$ (jako v \eqref{e:magnetizace v rovine})
\begin{equation}
    \left( \frac{\textrm{d}\vec{M}}{\textrm{d}\vHext} \right) \text{d}\vHext=\begin{pmatrix}
    -\sin \phim \\ \cos \phim \\ 0
    \end{pmatrix} \frac{\textrm{d}\phim}{\textrm{d}\phih}  \textrm{d}\phih \,,
\end{equation}
dostáváme
\begin{equation} \label{e:M integracni konstanta dodatek}
\mu_0\Hext M_S \int_{0}^{2\pi}  \frac{\textrm{d}\phim}{\textrm{d}\phih} \sin\left(\phih-\phim\right) \textrm{d}\phih=0 \,.
\end{equation}

\subsection*{Přibližný výpočet $\phim$}

\subsection*{Určení volné energie z nepřesného pole}
