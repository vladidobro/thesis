\section{Měření v transmisní geometrii: CoFe}
\label{chap:vysledky-cofe}

\subsection{Měření při pokojové teplotě}
\label{chap:vysledky-cofe-roomt}

Měření v transmisní geometrii při pokojové teplotě je nejjednodušší, protože nevyžaduje žádných zrcadel mezi vzorkem a můstkem.
Schéma aparatury je znázorněno na obr. \ref{fig:vysledky-cofe-schema-data} (a).
Příklad změřených dat a proloženého fitovaného modelu je na obr. \ref{fig:vysledky-cofe-schema-data} (b).
Výsledky fitu MO koeficientů $P$ jsou na obr. \ref{fig:vysledky-cofe-PMLD}.

Znázornění výsledků fitu magnetické anizotropie je na obr. \ref{fig:vysledky-cofe-roomt-anizotropie}.
\todo{napsat výsledky anizotropie}

\begin{figure}[htbp]
    \centering
    \missingfigure{}
    \caption{(a) schema (b) data}
    \label{fig:vysledky-cofe-schema-data}
\end{figure}

\begin{figure}[htbp]
    \centering
    \missingfigure{}
    \caption{PMLD cofe roomt, lowt}
    \label{fig:vysledky-cofe-PMLD}
\end{figure}

\begin{figure}[htbp]
    \centering
    \missingfigure{}
    \caption{}
    \label{fig:vysledky-cofe-roomt-anizotropie}
\end{figure}

\subsection{Měření při kryogenní teplotě}
\label{chap:vysledky-cofe-lowt}

Měření proběhlo při nejnižší možné teplotě dosažitelné použitým kryostatem, tj. přibližně \SI{15}{\kelvin}.
Měření v transmisní geometrii je při zchlazeném kryostatu výrazně složitější, viz obr. \ref{fig:vysledky-cofe-lowt-schema-data} (a).
První ze zrcadel je umístěno uvnitř komory kryostatu, druhé -- kompenzační -- je venku.
Není proto apriori zřejmé, zda se zrcadla skutečně kompenzují, když jsou udržovány na tak rozdílných teplotách.
Překvapivě tomu tak ale je: i při zběžném nastavení zrcadel byla po obou odrazech naměřena v celém spektru elipticita $<\SI{1}{\degree}$ pro vstupní polarizaci $\beta=45$, \SI{135}{\degree} (rovnoměrný mix s- a p-).

Druhý problém činí okénka komory kryostatu, na rozdíl od reflexní geometrie totiž světlo prochází postranním okénkem, ve kterém se indukuje Faradayův jev.
Důsledkem je pak měřené stočení téměř nezávislé na $\beta$, viz obr. \ref{fig:vysledky-cofe-lowt-schema-data} (b).
Provedli jsme pokus o oddělení Faradayova jevu od MLD na základě jejich symetrie.
Faradayův jev je lichý v magnetizaci (okénka) a je nezávislý na natočení vstupní lineární polarizace\footnote{V aparatuře se okénko vyskytuje až za prvním zrcadlem, takže není striktně pravda, že by stočení okénkem bylo nezávislé na $\beta$. Je nezávislé na natočení lineární polarizace vstupující do okénka.}.
Stočení po symetrizaci ve vnějším poli a vybrání $\beta$-závislosti s frekvencí $2\beta$ je na obr. \ref{fig:vysledky-cofe-lowt-schema-data} (c).
Signál je posléze fitovaný modelem popsaným v kap. \ref{chap:4} a poměrně překvapivě dává přibližně stejnou magnetickou anizotropii pro většinu vlnových délek, viz obr. \ref{fig:vysledky-cofe-lowt-schema-data} (d).
Právě kvůli Faradayově jevu v okénkách považujeme transmisní geometrii při nízké teplotě za nejméně spolehlivou.

Spektrální závislost výsledných MO koeficientů je vynesena na obr. \ref{fig:vysledky-cofe-PMLD}, společně s těmi měřenými při pokojové teplotě.

\begin{figure}[htbp]
    \centering
    \missingfigure{}
    \caption{(a) schema, (b) data puvodni, (c) data symetrizovana (d) mag anizotropie}
    \label{fig:vysledky-cofe-lowt-schema-data}
\end{figure}

