Funkčnost metody jsme ověřili jak v transmisní, tak reflexní geometrii.
Za zásadní nesporný důkaz funkčnosti považujeme určení magnetické anizotropie.
Výsledky získané různými vlnovými délkami jsou ve velmi dobré shodě, což je vzhledem k spektrální rozmanitosti MO koeficientů velice přesvědčivé: pokud bychom nepozorovali přímo signál od vzorku způsobený kvadratickým MO jevem, musel byl dodatečný nežádoucí signál mít stejnou spektrální závislost.
To je nepravděpodobné.

Na druhou stranu se ale vyskytuje problém zmíněný v oddílu \ref{chap:urceni-magneticke-anizotropie}, že anizotropie změřené různou velikostí pole $\Hext$ či různém natočení vzorku jsou odlišné.
Viz diskuze v příslušném oddílu.
Pokud jde skutečně o problém magnetu, jsou jistě ovlivněna i měření CoFe, ve kterých jsme problémy přímo nepozorovali.
Proto je třeba brát uvedené výsledky magnetické anizotropie s rezervou, uvádíme je především pro ilustraci spolehlivosti metody.

Uvádíme pouze finální sady měření.
Pokud byla použita zrcadla mezi vzorkem a detekcí, byla vždy kompenzována druhým zkříženým zrcadlem (viz oddíl \ref{chap:kompenzace}).

Na závěr je konečně metoda použita ke studiu antiferomagnetické fáze FeRh. 
V širší míře jsme zopakovali (tentokrát už s lepším porozuměním)\todopn{ok?} měření fázového přechodu feromagnet--antiferomagnet, provedené dříve v \cite{kubascikMagnetooptickeStudiumAntiferomagnetickych2019}.
