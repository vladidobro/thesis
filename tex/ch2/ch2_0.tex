V této kapitole se zaměříme na magnetooptické (MO) jevy -- jakým způsobem se tenzor permitivity závislý na magnetickém stavu projevuje v experimentu.
Změna permitivity se přirozeně projeví změnou transmisních a reflexních koeficientů, a tím pádem i v polarizačním stavu prošlého a odraženého světla.
Typický magnetooptický experiment tedy zkoumá polarizační stav světla po průchodu či odrazu od magnetooptického materiálu, měřenou veličinou je obvykle buď úhel natočení hlavní roviny polarizace $\beta$, elipticita $\chi$ nebo méně často intenzita $I$, v závislosti na magnetickém stavu.
Magnetooptické experimenty mají společné to, že u měřené veličiny většinou nelze prakticky určit absolutní hodnotu a je nutné měřit pouze rozdíly\cite{silberQuadraticMagnetoopticKerr2019a}.
Navíc se tato absolutní hodnota mění při změně většiny parametrů experimentu (např. posunutí/otočení čehokoliv), a jako nejčastější z mála možností je měřit rozdíl signálu při různých magnetických stavech.
To se realizuje např. změnou přiloženého vnějšího pole, současným pozorováním regionů v různých magnetických stavech (MO mikroskopie), nebo časově rozlišeným pozorováním po aplikaci silného krátkého laserového pulzu (pump-probe metody).

Jediná podmínka na zkoumaný materiál je, aby existoval v různých magnetických stavech.
Ty jsou nejčastěji charakterizované magnetizací $\vec{M}$ -- MO jevy byly pozorovány v diamagnetikách, paramagnetikách, feromagnetikách --
není to ale podmínkou, existují i v kompenzovaných antiferomagnetech (AFM)\cite{saidlOpticalDeterminationNeel2017}, ve kterých celková magnetizace vymizí.

MO jevy lze pozorovat v transmisi i v reflexi.
Reflexní jevy se souhrně nazývají MO Kerrovy jevy (MOKE) a většinou jsou slabší než transmisní\todocite{?}.
Pro Kerrovy jevy existuje ustálená notace výsledků měření\cite{silberQuadraticMagnetoopticKerr2019a}.
Kvůli praktickým účelům bývá nenulový úhel dopadu a jako vstupní polarizace se zpravidla volí buď s- nebo p-polarizace. 
Změna polarizačního stavu po odrazu je popsána tzv. Kerrovou rotací a Kerrovou elipticitou vyjádřených společně komplexní Kerrovou rotací $\Psi_{s/p}$ jako v \eqref{eqn:komplexni-rotace}.

Dalším přirozeným dělením MO jevů je třídění na liché a sudé v magnetizaci, což vzhledem k běžnému zanedbání jevů třetího a vyššího řádu splývá s dělením na lineární a kvadratické.
Tradičně však toto dělení nerozlišuje magnetickou závislost permitivity (vyjádřenou MO tenzory $\mathbb{K}$ a $\mathbb{G}$), ale pozorovaných veličin (např. $\beta$, $\chi$, příp. transmisní/reflexní koeficienty). 
Závislost měřitelných veličin na $\varepsilon$ je nelineární, a proto mohou být kvadratické jevy způsobeny i lineárním $\mathcal{K}$: např. v kvadratickém Voigtově jevu (viz níže) vystupuje i člen $\left(\varepsilon^{(1)}\right)^2 \propto \mathbb{K}^2 \vec{M}^2$, který je někdy označovaný jako \emph{beam walk-off}\cite{akbarLowTemperatureVoigt2017}.
Lineární jevy mohou naopak být způsobeny jedině lineárním $\mathbb{K}$.

Zmíníme ještě jedno dělení MO jevů: na základě toho, jaký druh optické anizotropie do vzorku zavádí magnetická permitivita\cite{zvezdinModernMagnetoopticsMagnetooptical1997}.
Anizotropie může mít charakter dvojlomu (anizotropie $\Re\left\lbrace n\right\rbrace$ -- rozdílná fázová rychlost) nebo dichroismu (anizotropie $\Im\left\lbrace n \right\rbrace$ -- rozdílný absorpční koeficient).
Druhou charakteristikou optické anizotropie jsou vlastní módy, pro které se daná vlastnost liší -- v nejjednodušším případě jsou to lineární\footnote{Slovo lineární má zde význam lineární polarizace; nesouvisí s dělením na lineární a kvadratické jevy výše. MLD je kvadratický jev.} nebo kruhové polarizace.
Kombinací těchto charakteristik dělíme jevy na magnetický lineární/kruhový dvojlom/dichroismus: MLB, MLD, MCB, MCD\footnote{angl. Magnetic Linear/Circular Birefringence/Dichroism.}.
Toto dělení se ale v reflexi používá z ``pedantského'' pohledu chybným způsobem: jako MLD se někdy označuje anizotropie reálné části \emph{reflexního koeficientu}, která je však spjatá s anizotropií reálné i imaginární části indexu lomu -- tj. MLB i MLD\cite{tesarovaSystematicStudyMagnetic2014}.

Poznamenáme, že pro intuitivní a dokonce i kvantitativní pochopení vlivu průchodu světla vzorkem vykazujícím jednu z uvedených anizotropií je dostačující grafické znázornění Muellerovych matic zavedené v oddílu \ref{chap:Stokes-Mueller}.
Dvojlomný vzorek (obecný retardér) provádí otáčení a dichroický (obecný diatenuátor) protahování a posouvání ve směru osy procházející příslušnými lineárními/kruhovými módy.

\paragraph{Lineární jevy}

Jsou obecně silnější a navíc mají tu výhodu, že díky opačným znaménkám signálu pro opačné směry magnetizace lze často jednoduše oddělit signál od nemagneto-optického pozadí.
V reflexi se lineární Kerrovy jevy rozlišují podle toho, na jaké složky magnetizace jsou citlivé, viz obr. \ref{fig:MOKE-Silber}\cite{silberQuadraticMagnetoopticKerr2019a}.
\begin{itemize}
    \item Polární (PMOKE) - $\vec{M}$ kolmé na rozhraní.
    \item Longitudinální (LMOKE) - $\vec{M}$ v rovině rozhraní a rovině dopadu.
    \item Transverzální (TMOKE) - $\vec{M}$ v rovině rozhraní a kolmo na rovinu dopadu.
\end{itemize}
LMOKE a TMOKE jsou nulové při kolmém dopadu, a při téměř kolmém dopadu jejich amplituda roste lineárně s úhlem dopadu.
PMOKE je nenulový i při kolmém dopadu.

V transmisi se vyskytují podobné jevy, z nichž samostatný název má jako jediný polární - Faradayův jev (MCB) objevený jako první už v roce 1845, ještě před formulací Maxwellových rovnic\cite{zvezdinModernMagnetoopticsMagnetooptical1997}.

\begin{figure}[htbp]
    \centering
    \includegraphics[width=\textwidth]{./img/static/linMOKE-silber.pdf}
    \caption{Lineární Kerrovy jevy. Předposlední řádek uvádí tvar permitivity $\varepsilon$ izotropního materiálu s danou magnetizací. Poslední řádek výslednou reflexní Jonesovu matici. \cite{silberQuadraticMagnetoopticKerr2019a}}
    \label{fig:MOKE-Silber}
\end{figure}

\paragraph{Kvadratické jevy}

Zatímco lineární magnetooptika se stala užitečným nástrojem v mnoha oborech, kvadratická dlouho unikala pozornosti.
Prakticky jediná situace, ve které je díky vymizení lineárních jevů možné pozorovat čistě kvadratické jevy, je kolmý dopad s transverzální magnetizací\footnote{Nebo nulový $\mathbb{K}$ tenzor jako v případě kolineárních AFM.}, jak znázorňuje obr. \ref{fig:Voigtova-geometrie}.
Poprvé byl kvadratický jev pozorován v transmisi -- Voigt v roce 1902 pozoroval v plynech stočení kvadraticky závislé na transverzálním magnetickém poli a v roce 1907 to samé v kapalinách nezávisle pozorovali Cotton a Mouton\cite{zvezdinModernMagnetoopticsMagnetooptical1997}.
Z pohledu třídění magneto-optických anizotropií se jedná o MLD, jev se často nazývá Voigtův či Cottonův-Moutonův.
V reflexi se kvadratické jevy nazývají souhrnně jako kvadratické MOKE (QMOKE), případně reflexní Voigtův, Cotton-Moutonův jev nebo reflexní analogie MLD.

Transmisní i reflexní verze tohoto jevu jsou ústředním tématem této práce a pro jednoduchost je dále budeme nazývat souhrnně MLD.

\begin{figure}[htbp]
    \centering
    \includegraphics{./img/static/voigt-geometry-tesarova.pdf}
    \caption{Voigtova geometrie. Světlo dopadá kolmo na rozhraní a magnetizace je transverzální (v rovině rozhraní). \cite{tesarovaSystematicStudyMagnetic2014}}
    \label{fig:Voigtova-geometrie}
\end{figure}

Kvadratické jevy se postupem času ukázaly jako obecně téměř všudypřítomné a nezanedbatelné.
V roce 2005 byl v magnetickém polovodiči \ch{GaMnAs} pozorován v reflexi při téměř kolmém dopadu \emph{obří} MLD - kvadratické stočení polarizace, které bylo svou velikostí srovnatelné s lineárními jevy\cite{kimelObservationGiantMagnetic2005}.
K pozorování bylo využito chování hysterezních smyček v materiálech se čtyřmi snadnými osami, viz. obr. \ref{fig:obri-MLD}.

\begin{figure}[htbp]
    \centering
    \includegraphics{./img/static/mld-hystereze-kimel.pdf}
    \caption{Obří MLD v GaMnAs. (a) Snadné osy magnetizace označené čísly (1)--(4). (b) Stočení polarizace v při aplikaci vnějšího pole ve směru $\theta$. Spektrální závislost (c) síly MO jevů a (d) absorpce. \cite{kimelObservationGiantMagnetic2005}}
    \label{fig:obri-MLD}
\end{figure}

Silné MLB a MLD bylo také pozorováno při nízkých teplotách v paramagnetickém terbium-galiovém granátu (TGG)\cite{akbarLowTemperatureVoigt2017} a obecně jsou často silné v Heuslerových sloučeninách\cite{hamrleHugeQuadraticMagnetooptical2007}.
V roce 2020 byl pozorován obří QMOKE v tenkém filmu \ch{(Eu,Gd)O} dosahující stočení až \SI{1}{\degree}\cite{katsGiantQuadraticMagnetooptical2020}.

Jedním z důvodů, proč se v současné době kvadratické jevy těší vysoké popularitě, je, že kvadratický tenzor $\mathbb{G}$ dovoluje výrazně širší třídu symetrií materiálů než lineární $\mathbb{K}$; mezi ně se řadí např. zmíněné kolineární AFM. 
AFM byly již dříve předmětem intenzivního spintronického výzkumu \cite{nemecAntiferromagneticOptospintronics2018,jungwirthAntiferromagneticSpintronics2016}, 
s objevem nové třídy magnetických materiálů -- \emph{altermagnetů}\cite{smejkalAltermagnetismSpinmomentumLocked2021} -- se opět dostávají na výsluní.

Co se týče praktického použití kvadratických jevů pro studium AFM, byly využity např. pro určení Néelova vektoru v \ch{CuMnAs}\cite{saidlOpticalDeterminationNeel2017}, pro mikroskopii AFM domén\cite{xuImagingAntiferromagneticDomains2019} tenkých filmů \ch{NiO} (viz obr. \ref{fig:AFM-mikroskopie-domen}) a pro pozorování reakcí \ch{Fe2As} na ultrarychlé změny teploty\cite{yangMagnetoopticResponseMetallic2019}.

\begin{figure}[htbp]
    \centering
    \includegraphics{./img/static/AFM-mikroskopie-xu.pdf}
    \caption{Pozorování antiferomagnetických domén v tenkém filmu NiO. (a) Schéma experimentu: je zkoumán rozdíl stočení polarizace při odrazu od domén s vzájemně kolmou orientací Néelova vektoru. (b--e) Závislost pozorovaného odrazu na natočení vstupní lineární polarizace. \cite{xuImagingAntiferromagneticDomains2019}}
    \label{fig:AFM-mikroskopie-domen}
\end{figure}

V nejjednodušším případě Voigtovy geometrie, kdy má navíc magnetická závislost permitivity plnou symetrii prázdného prostoru ($\mathbb{K}$, $\mathbb{G}$ i všechny vyšší řády jsou izotropní), má stočení polarizace vlivem Voigtova jevu jednoduchý tvar s fenomenologickým parametrem $\Pmld$ popisujícím amplitudu jevu\cite{tesarovaHighPrecisionMagnetic2012}
\begin{equation} 
\label{eqn:PMLD}
    \Delta \beta=\Pmld \sin\left(2(\phim-\beta)\right) \,.
\end{equation}

Ani kubické krystaly však obecně nemají izotropní $\mathbb{G}$\cite{hamrlovaQuadraticinmagnetizationPermittivityConductivity2013}, a navíc pro měření odrazu se často používá malý, ale nenulový úhel dopadu, který vnese do signálu i lineární MOKE, takže je třeba využít plného formalismu představeného v kap. \ref{chap:1}.
To je sice v principu možné pro každou konkrétní situaci, ale pro interpretaci experimentu je nutné mít nějaký fenomenologický vzorec typu \eqref{eqn:PMLD}, který je platný pro širší třídu situací.

Šikmý odraz na polonekonečném (bulku) [001] orientovaném kubickém vzorku s omezením na in-plane magnetizaci byl spočítán pomocí Yehovy metody v \cite{postavaAnisotropyQuadraticMagnetooptic2002}.
Jiná situace byla spočítána Berremanovou metodou v \cite{hamrleVicinalInterfaceSensitive2003}: odraz na struktuře tvořené izotropním polonekonečným substrátem,
ultra-tenkým\footnote{$nd/k_0 \ll 1$, kde $n$ je index lomu a $d$ tloušťka vrstvy.}
filmem se zcela obecně anizotropním tenzorem permitivity, a izotropní nadvrstvou.
Dosazením magnetické permitivity pro [001] orientovaný kubický vzorek in-plane otočený o úhel $\gamma$, jako v \eqref{eqn:funkcional-otoceny}, je pak reprodukován stejný tvar vzorce jako pro polonekonečný vzorek
\begin{align} 
\label{eqn:QMOKE-vzorec}
    \Psi_{s/p}=
    & A_{s/p} \left[ 2G_{44}-\frac{K^2}{\varepsilon^0}+\frac{\Delta G}{2}(1-\cos 4\gamma)  \right] M_x M_y\\
    & + A_{s/p} \left[ \frac{\Delta G}{4}\sin 4\gamma  \right] \left(M_x^2-M_y^2\right) \pm B_{s/p} K M_y \,,
\end{align}
kde $A_{s/p}$ a $B_{s/p}$ jsou vážící konstanty závisející na úhlu dopadu a na parametrech substrátu a nadvrstvy. 
$A_{s/p}$ je sudou a $B_{s/p}$ lichou funkcí úhlu dopadu, jejich konkrétní tvar je uveden v příslušných původních článcích\cite{postavaAnisotropyQuadraticMagnetooptic2002,hamrleVicinalInterfaceSensitive2003}.
Detaily ohledně konvencí lze také nalézt v \cite{silberQuadraticMagnetoopticKerr2019a}.
Vliv izotropního substrátu a nadvrstvy je diskutován metodou efektivních rozhraní v \cite{visnovskyPolarMagnetoopticsSimple1995}. 

V uvedeném vzorci \eqref{eqn:QMOKE-vzorec} jsou patrné dva druhy příspěvků: lineární v $\vec{M}$, který vymizí při kolmém dopadu a kvadratický v $\vec{M}$, který je nenulový i při kolmém dopadu.
Jak bylo avizováno, kvadratický člen v sobě zahrnuje i lineární permitivitu $K^2$.
Lze také vidět, že při kolmém dopadu $B_{s/p}=0$ a izotropním $\mathbb{G}$ ($\Delta G=0$) nezáleží signál na natočení vzorku a dostaneme
\begin{equation}
    \Psi_{s/p}=A_{s/p} \left[ 2G_{44}-\frac{K^2}{\varepsilon^0}  \right] M_x M_y \,,
\end{equation}
což je ekvivalentní \eqref{eqn:PMLD}.

Při nekolmém dopadu se lineární i kvadratický člen sčítají a pro kvantitativní analýzu je třeba je nějakým způsobem oddělit.
Metodám, které se tím zabývají, se věnujeme v dále uvedených částech této kapitoly.
