Jediným cílem této práce je zprovoznění a zdokonalení jednoho druhu magneto-optického experimentu na pracovišti Laboratoři OptoSpintroniky (LOS) KChFO MFF UK na adrese Ke Karlovu 3, Praha 2.\todopn{je to dobře napsané? měl bych zmínit FÚ?}

Experiment lze nejvýstižněji popsat jako \emph{spektroskopii anizotropního MLD} (/Voigtova/Cotton-Moutonova jevu) a přidruženou \emph{in-plane magneto-metrii} v tenkých filmech, kolmém dopadu a rotujícím poli.
Základní schéma aparatury je na obr. \ref{fig:zakladni-schema}.
Následuje popis jednotlivých elementů

\begin{figure}[htbp]
    \centering
    \missingfigure{schema}
    \caption{schema}
    \label{fig:zakladni-schema}
\end{figure}

\paragraph{2-D Magnet}
Elektro-magnet tvořený dvěma páry nezávislých cívek dokáže vytvořit libovolné vnější magnetické pole $\vHext$ v rovině $xy$ o maximální velikosti \SI{210}{\milli\tesla}.
V praxi jsou kvůli hysterezi magnetu použity pouze definované charakterizované procedury (posloupnosti proudů).
Vývojem proudových tabulek se zabývá \cite{kimakCharakterizaciaDvojdimenzionalnehoElektromagnetu2017,kimakOptickaSpektroskopieAntiferomagnetu2019}.
V této práci používáme pouze dvě: rotaci pole o velikosti $\Hext=\SI{207}{\milli\tesla}, \SI{50}{\milli\tesla}$ s krokem v úhlu pole $\phih$ minimálně \SI{1}{\degree}.

\paragraph{Kryostat}
Kryostat s topením dovoluje udržovat vzorek v rozmezí teplot cca 15--\SI{800}{\kelvin}.
Vzorek je lepen na \emph{cold-finger} a v kryogenické komoře umístěn mezi ramena magnetu.
Komora je opatřena zpředu a ze stran skleněnými okénky.
Magneto-optickou aktivitu okének (Faradayův jev) zkoumá \cite{baduraMagnetooptickaMereniPro2019}.

\paragraph{Super-kontinuální laser}
\emph{SuperK EXTREME} generuje široko-spektrální pulzy, které jsou dále filtrovány pro získání pulzů s šířkou pásma \SI{10}{\nano\meter}.
V rozmezí \num{460}--\SI{845}{\nano\meter} k filtraci používáme laditelný filtr \emph{SuperK VARIA}, v rozmezí \num{845}--\SI{1600}{\nano\meter} pak sadu pásmových interferenčních filtrů.

\paragraph{Polarizační optika}
Polarizatóry P1 a P2 jsou širokospektrální typu Glan Laser.

\subparagraph{PEM}
\cite{minarModulacePolarizaceSvetelne2004}
\subparagraph{Berekův kompenzátor}
\cite{schusserSkryteKouzloPolarizace2014}

\paragraph{Detektory}
\cite{hovorakovaCharakterizaceInfracervenehoDetektoru}
