\section{Detekce}
\label{chap:detekce}

V tomto oddílu se zaměříme na první z problémů, který je způsobený tím, že v uspořádání na obr. \ref{fig:zakladni-schema} neměříme přímo stočení polarizace vzorkem, tj. neplatí vzorec \eqref{eqn:mustek-delta-beta}: $\Udif/2\Usum\neq\Delta\beta$.
Příčiny jsou dvě: zrcadla mezi vzorkem a můstkem měnící polarizaci, a nedokonalost vlnové destičky a děliče v můstku.

Nejdříve popíšeme optický můstek pomocí Stokesova-Muellerova formalismu a grafického znázornění zobrazováním Poincarého sféry (viz oddíl \ref{chap:Stokes-Mueller}).
Dále vysvětlíme podstatu problémů a popíšeme několik způsobů, jakým vliv zrcadel a nedokonalostí kompenzovat.
Nakonec v oddílu \ref{chap:elipticita} popíšeme způsob využití Berekova kompenzátoru v můstku, který umožňuje současné měření stočení i elipticity.

\subsection{Stokesovy kovektory detektorových systémů}

Popis pomocí Stokesových vektorů je v našem případě zvláště užitečný, protože Stokesovy parametry jsou \emph{zobecněné intenzity}.
Intenzita měřená detektorem A je daná nultou složkou dopadajícího Stokesova vektoru
\begin{equation}
    I_\textrm{A}=s_0^\textrm{A} \equiv \Dtks^\textrm{A} \cdot \Stks^\textrm{A} \,.
\end{equation}
kde jsme označili $\Dtks^\textrm{A}=(1, 0, 0, 0)$ ``Stokesův kovektor`` detektoru A. 
Tečka ($\cdot$) značí kontrakci\footnote{``Skalární součin''.}.

Vyjádřením $\Stks^\textrm{A}$ pomocí Stokesova vektoru vstupujícího do můstku $\Stks^\textrm{in}$ a Muellerových matic půlvlnné destičky a polarizačního děliče
\begin{equation}
    I_\textrm{A}= \Dtks^\textrm{A} \cdot \left( \M_\textrm{A} \M_{\lambda/2} \Stks^\textrm{in} \right) = \left( \Dtks^\textrm{A} \M_\textrm{A} \M_{\lambda/2}\right) \cdot \Stks^\textrm{in} \equiv {\Dtks'}^\textrm{A} \cdot \Stks^\textrm{in}  \,,
\end{equation}
kde jsme druhou rovností naznačili asociativitu maticového násobení, která nás motivovala k definici Stokesova kovektoru detektoru A vzhledem ke světlu vstupujícímu do můstku ${\Dtks'}^\textrm{A}$.

Stejně lze zřejmě činit i pro detektor B.
Rozdílové a součtové napětí je pak
\begin{align}
\label{eqn:af}
\Udif = \left( {\Dtks'}^\textrm{A}-{\Dtks'}^\textrm{B}\right) \cdot \Stks^\textrm{in} \equiv {\Dtks}^\textrm{A-B}\cdot \Stks^\textrm{in} \,, \\
\Usum = \left({\Dtks'}^\textrm{A}+{\Dtks'}^\textrm{B}\right) \cdot \Stks^\textrm{in} \equiv {\Dtks}^\textrm{A+B}\cdot \Stks^\textrm{in} \,.
\end{align}

Měřené signály jsou lineární ve vstupních Stokesových vektorech a jsou tedy reprezentovány lineárními formamy, zde ${\Dtks''}^\textrm{A-B}$ a ${\Dtks''}^\textrm{A+B}$.
Pro ideální prvky jako v oddílu \ref{chap:mustek-kap2} jsou Muellerovy matice
\begin{align}
    \M_\textrm{A} = \begin{pmatrix} 1&1&0&0 \\ 1&1&0&0 \\ 0&0&0&0 \\ 0&0&0&0 \end{pmatrix} \,,\quad 
    \M_\textrm{B} = \begin{pmatrix} 1&-1&0&0 \\ -1&1&0&0 \\ 0&0&0&0 \\ 0&0&0&0 \end{pmatrix} \,, \\
    \M_{\lambda/2} = \begin{pmatrix} 0&0&0&0 \\ 0&\cos\theta_{\lambda/2}&\cos\theta_{\lambda/2}&0 \\ 0&\cos\theta_{\lambda/2}&\cos\theta_{\lambda/2}&0 \\ 0&0&0&1 \end{pmatrix}
\end{align}
a Stokesovy kovektory jsou po dosazení
\begin{align}
    \Dtks^\textrm{A-B}=(0, \cos\theta_{\lambda/2}, \cos\theta_{\lambda/2}, 0) \,, \label{eqn:Uab-Mueller} \\
    \Dtks^\textrm{A+B}=(1, 0, 0, 0) \,.
\end{align}


Místo sledování, co se při průchodu optickými prvky děje se všemi myslitelnými vstupními Stokesovými vektory, je výhodnější sledovat, jak se mění konstantní detektorové kovektory.
Navíc Stokesovy kovektory lze pro plně polarizováné světlo přímočaře graficky znázorňovat v 3-D prostoru redukovaných Stokesových vektorů.
Pokud je jediná nenulová složka kovektoru $d_0$ ta nultá (jako v $\Dtks^\textrm{A+B}$), pak je měřený signál jednoduše úměrný vstupní intenzitě (vzdálenosti redukovaného Stokesova vektoru od počátku) a netřeba ho zvlášť znázorňovat.
Pokud je naopak $d_0=0$ (jako v $\Dtks^\textrm{A+B}$), pak je signál lineární v redukovaném Stokesově vektoru a tedy úměrný vzdálenosti od určité roviny.

V obecném případě, tj. $d_0$ není ani nulová, ani jediná nenulová, je znázornění mírně složitější.
Nejedná se o pouhé posunutí roviny, od které je vzdálenost odečítána.
Na obr. \ref{fig:mustek-znazorneni-kovektoru} (a) jsou vyznačené nulové plochy (množina Stokesových vektorů, pro které $\Udif=0$) $\Dtks^\textrm{A-B}$ tak, jak je v \eqref{eqn:Uab-Mueller} a v situaci, kdy $d_0$ je nenulové.
Pro nenulové $d_0$ je nulová plocha kužel.
Pro body neležící na nulovém kuželu však $\Udif$ není určeno vzdáleností od kužele.
Pro body na Poincarého sféře je signál úměrný vzdálenosti od roviny, která vznikne průnikem nulového kuželu s Poincarého sférou (vždy kružnice); stejnou konstrukci lze provést i pro body mimo Poincarého sféru ($I\neq1$), posunutí roviny však závisí na $I$.

Uvedený postup selhává, pokud $d_0>|\vec{d}|\equiv|(d_1, d_2, d_3)|$, protože pak neexistují nenulové vektory, pro které je signál nulový; neexistuje nulová plocha.
Nicméně pro každé konstantní $I$ je stále signál úměrný vzdálenosti od určité roviny, tentokrát však ležící mimo sféru vektorů s intenzitou $I$.

Z provedené diskuze vyplývá jako nejvýhodnější znázorňovat Stokesovy kovektory jednotně pomocí \emph{nulové kružnice}, která je průsečíkem nulového kužele s Poincarého sférou, a jejím normálovým 3-vektorem $\vec{d}$, kterým je znázorněna konstanta úměrnosti mezi signálem a grafickou vzdáleností.
Případ $d_0>|\vec{d}$ v této práci nepoužíváme (formalismus používáme pro popis optického můstku, ve kterém se takové kovektory nevyskytují), ale znázorňovali bychom ho také 3-vektorem $\vec{d}$ a ``nulovou''\footnote{Která vzhledem k tomu, že leží mimo Poincarého sféru, odpovídá nedosažitelným Stokesovým vektorům $|\vec{s}|>s_0$.} rovinou příslušnou Poincarého sféře.
Viz obr. \ref{fig:mustek-znazorneni-kovektoru} (b).

\begin{figure}[htbp]
    \centering
    \missingfigure{kovektory}
    \caption{(a) Nulové plochy Stokesových kovektorů s nulovou/nenulovou nultou složkou. (b) Kanonické znázornění stejných kovektorů způsobem, který je výhradně použit dále v této práci.}
    \label{fig:mustek-znazorneni-kovektoru}
\end{figure}

Při znázornění kovektorů tímto způsobem lze beze změny převzít znázornění akce Muellerových matic pomocí mapování Poincarého sféry.
Protože nulové kružnice jsou množiny Stokesových vektorů, je možné je přímo zobrazit metodou vyloženou v oddílu \ref{chap:Stokes-Mueller} a výsledkem bude opět podmnožina nulové plochy.
V případě čistých retardérů se nulová kružnice zobrazí jako jiná kružnice na Poincarého sféře, a je to proto přímo nulová kružnice transformovaného kovektoru; vektor $\vec{d}$ se otáčí společně s kružnicí.
Pokud však Muellerovou maticí dochází také k deformaci a posunutí Poincarého sféry (jako např. v případě čistého diatenuátoru), je obrazem zobrazení obecně elipsa ležící uvnitř Poincarého sféry, a je proto nutné provést následující konstrukci pro získání nulové kružnice.
Transformovaná elipsa vždy leží na nulovém kuželu, takže ho můžeme určit proložením elipsy kuželem se vrcholem v počátku.
Nulová kružnice je pak určena průnikem kuželu a Poincarého sféry.
Konstrukce je znázorněna na obr. \ref{fig:kovektor-akce-H}.\todo{mozna jeste co se deje s vec d}

\begin{figure}[htbp]
    \centering
    \missingfigure{konstrukce H kruznice}
    \caption{Transformace Stokesova kovektoru neunitárním optickým prvkem. Muellerova matice deformuje Poincarého sféru na elipsoid, takže nulová kružnice se zobrazí na elipsu uvnitř sféry. Elipsou je proložen kužel a průnikem se sférou je určena nová transformovaná nulová kružnice.}
    \label{fig:kovektor-akce-H}
\end{figure}


\subsection{Kompenzace nedokonalostí a zrcadel}
Jedna z výhod formalismu Stokesových kovektorů snadné uvažování o tom, jak se detekční aparatura chová při malých změnách (nedokonalostech) optických prvků.
Zaměříme se na tři druhy druhy nedokonalostí, z nichž se nakonec jediná vyplatí kompenzovat -- nepřesné fázové zpoždění půlvlnné destičky.

Pro vyvažovací půlvlnnou destičku uvažujeme dva druhy nedokonalosti: rozdílnou propustnost obou módů a fázové zpoždění lišící se od přesné hodnoty $\pi/2$.
Kvůli symetrii však požadujeme, aby destička měla dvě navzájem kolmé optické osy -- dva vlastní módy ortogonálních lineárních polarizací.

Experimentálně bylo ověřeno, že polarizační dělič vysoce kvalitně dělí svazek na dvě ortogonální lineární polarizace.
V odraženém svazku je sice zastoupeny obě polarizace, ale kvůli mírně odlišnému úhlu lomu se prostorově oddělí.
Vložením polarizátoru před dělič a jeho vhodným otáčením bylo možné ho zkřížit vzhledem k oběma ramenům (zvlášť) s extinkčním poměrem $I_\textrm{min}/I_\textrm{max} \approx \num{1e-4}$.
Díky tomu se nijak neprojeví ani případná polarizační závislost detektorů.
Jediná nedokonalost zbytku můstku (nezahrnující půlvlnnou destičku) je tedy vyjádřena rozdílnou citlivostí obou ramen na příslušné lineární polarizace, která je způsobena jak rozdílnou propustností/odrazivostí děliče, tak nevyváženou citlivostí a zesílení obou detektorů.

Všechny tři zmíněné nedokonalosti uvažujeme zvlášť a zanedbáváme jejich vzájemné působení.
Nakonec se zaměříme na to, co se stane, když před můstek umístíme retardér -- zrcadlo logisticky nezbytné pro oddělení dopadajícího a odraženého svazku v reflexní geometrii, a pro vyvedení svazku ven z komory kryostatu v transmisní geometrii.

Pro další použití na obr. \ref{fig:kovektor-ideal-mustek} vykreslujeme kovektory ideálního můstku vzhledem ke světlu po (${\Dtks'}^\textrm{A-B}$) a před (${\Dtks''}^\textrm{A-B}$) průchodem půlvlnnou destičkou.

\begin{figure}[htbp]
    \centering
    \missingfigure{kovektory idealni}
    \caption{Stokesovy kovektory ideálního můstku. (a) Vzhledem ke světlu před děličem. (b) Vzhledem ke světlu před destičkou. (c) Ilustrace významu $\Dtks'$ a $\Dtks''$.}
    \label{fig:kovektor-ideal-mustek}
\end{figure}

\subsubsection*{Nevyváženost ramen}
Rozdíl citlivostí ramen je vyjádřen $\eta$, takže vzhledem ke světlu před děličem
\begin{align}
    {\Dtks'}^\textrm{A}=\frac{1+\eta}{2}(1, 1, 0, 0) \,,\\
    {\Dtks'}^\textrm{B}=\frac{1-\eta}{2}(1, -1, 0, 0) \,,\\
    {\Dtks'}^\textrm{A-B}=(\eta, 1, 0, 0) \,,\\
    {\Dtks'}^\textrm{A+B}=(1, \eta, 0, 0) 
\end{align}
a vzhledem ke světlu před destičkou
\begin{align}
    {\Dtks''}^\textrm{A-B}=(\eta, \cos(\theta_{\lambda/2}), \cos(\theta_{\lambda/2}), 0) \,,\\
    {\Dtks''}^\textrm{A+B}=(1, \eta\cos(\theta_{\lambda/2}), \eta\cos(\theta_{\lambda/2}), 0) 
\end{align}

Prvním z projevů je polarizační závislost součtového signálu ($d_1^\textrm{A+B}=\eta\neq=0$).
To není velký problém, protože se ve vzorci $\Delta\beta=\Udif/2\Usum$ používá pouze pro normalizaci signálu, která lze určit i jiným způsobem (viz např. oddíl \ref{chap:elipticita}). 
Pro malé $\eta$ však není od věci tento vliv ignorovat.

Druhou známkou nevyvážených ramen je nenulové $d_0^\textrm{A-B}$, které se projeví posunutím nulové kružnice, viz obr. \ref{fig:mustek-nedokonale-ramena} (a), po otočení půlvlnnou destičkou pak \ref{fig:mustek-nedokonale-ramena} (b).
Důležitým rysem je, že nulová kružnice je vždy kolmá na rovník lineárních polarizací, což znamená, že $\partial\Udif/\partial\chi = 0$, a tedy v měřeném signálu se neprojeví změny elipticity (, pokud do můstku dopadá lineárně polarizované světlo).
I druhý projev tedy v důsledku pouze mění konstantu úměrnosti, pro $\chi=0$ platí
\begin{equation}
    \textrm{d}\Udif = 2I\cos(\eta) \textrm{d}\beta \,,
\end{equation}
což pro malé $\eta$ ignorujeme.

\begin{figure}[htbp]
    \centering
    \missingfigure{nedokonale ramena}
    \caption{Kovektory můstku s nevyváženými rameny pro světlo (a) před děličem, (b) před destičkou.}
    \label{fig:mustek-nedokonale-ramena}
\end{figure}

\subsubsection*{Rozdílná propustnost destičky}
Jonesova a Muellerova matice destičky s přesným fázovým zpožděním $\pi/2$, ale rozdílnou propustností obou lineárních polarizací je
\begin{equation}
    \mathcal{T}_{\lambda/2} = \begin{pmatrix} 1&0 \\ 0&-\eta \end{pmatrix} \,, \quad
    \M_{\lambda/2} = \begin{pmatrix} 1&0&0&0 \\ 0&0&0&0 \\ 0&0&0&0 \\ 0&0&0&-1 \end{pmatrix} \,.
\end{equation}
Akce takové Muellerovy matice je kombinace otočení o \SI{180}{\degree} a protáhnutí/posunutí kolem stejné osy -- procházející lineárními polarizacemi ve směru optické osy $\theta_{\lambda/2}$.


\subsubsection*{Nepřesné fázové zpožední destičky}

\subsubsection*{Zrcadla}

\subsection{Současné měření elipticity}
\label{chap:elipticita}
