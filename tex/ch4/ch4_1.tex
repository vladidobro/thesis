\section{Výchozí situace}
\label{chap:vychozi-situace}

Aparaturou se již zabývali dřívější práce vzniklé na pracovišti.
\cite{kimakCharakterizaciaDvojdimenzionalnehoElektromagnetu2017,kimakOptickaSpektroskopieAntiferomagnetu2019} se věnují charakterizaci magnetu a implementaci měřících procedur.
\cite{wohlrathMagnetooptickaCharakterizaceSpintronickych2018} uvádí do provozu první verzi optické aparatury a provádí zkušební experimenty, \cite{kubascikMagnetooptickeStudiumAntiferomagnetickych2019,kimakOptickaSpektroskopieAntiferomagnetu2019} aparaturu vylepšují a využívají k magneto-optickým měřením AFM materiálů.

Žádné z výsledků předchozích prací však nebyly uspokojující.
Úvodním krokem této práce bylo zopakovat a rozšířit spektroskopická měření MLD ve vzorku FR06 ve FM fázi, provedená v \cite{kubascikMagnetooptickeStudiumAntiferomagnetickych2019}.
Měření byla úspěšně replikovaná, ale problémy přetrvaly.
