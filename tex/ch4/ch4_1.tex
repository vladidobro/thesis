\section{Popis a úvodní experimenty}
\label{chap:vychozi-situace}

Stejnou metodou (se stejnou aparaturou) se zabývaly i dřívější práce vzniklé na pracovišti.
První MO experimenty byly provedeny v \cite{wohlrathMagnetooptickaCharakterizaceSpintronickych2018}. 
Aparatura byla vylepšena a využita ke studiu AFM materiálů v \cite{kubascikMagnetooptickeStudiumAntiferomagnetickych2019,kimakOptickaSpektroskopieAntiferomagnetu2019}.

Dosavadní výsledky však nebyly uspokojující.
V rámci této práce byly pro ověření zopakována měření vzorku FR06 ve FM fázi v reflexní geometrii, provedena dříve v \cite{kubascikMagnetooptickeStudiumAntiferomagnetickych2019}.
Měření byla úspěšně replikovaná, ale problémy přetrvaly.
Jako druhá sada úvodních měření byl měřen vzorek CoFe v transmisní geometrii.

Průběh experimentu je následující.
Komora kryostatu je vyčerpána a udržována na požadované konstantní teplotě (kromě měření teplotních přechodů).
Některým filtračním prvkem je laser nastaven na požadovanou vlnovou délku a správným natočením polarizátoru P1 a půlvlnné destičky WP1 je nastavena lineární polarizace v požadovaném směru $\beta$.
Natočením WP2 je pak vyvážen optický můstek.

Následuje měřící procedura. 
Dvěma nezávislým ramenou elektromagnetu jsou postupně zadávány dvojice kalibrovaných proudů, které vytváří magnetické pole $\vHext$ o konstantní velikosti $\Hext$ (zde vždy \num{50} nebo \SI{207}{\milli\tesla}), a otáčí ho (diskrétními kroky) o \SI{360}{\degree} v rovině $xy$.
V každém kroku (daného směrem pole $\phih$) je zaznamenána hodnota rozdílového $\Udif$ a součtového $\Usum$ napětí na detektorech.
Signál je zpracován a s předpokladem idealizovaného můstku (viz oddíl \ref{chap:mustek-kap2}) dosazen do vzorce \eqref{eqn:A-B-mustek}.
Výsledkem je tedy průběh stočení polarizace $\Delta\beta(\phih)$.
V praxi měříme více cyklů, způsob jakým z měřených dat získat $\Delta\beta(\phih)$ je detailně popsán v dodatku \ref{app:zpracovani}.

Měřící procedura je pak opakována pro více $\beta$, $\lambda$ a případně $T$.
Při změně $\beta$ či $\lambda$ je však třeba znovu vyvažovat můstek, čímž se mění aditivní konstanta vyjadřující neznámé vyvážení můstku $\xi$.
Změřené $\Delta\beta$ pro různé $\beta$ a $\lambda$ tedy není možné přímo porovnávat.
Zatímco měření různých $\beta$ spolu úzce souvisí, různé $\lambda$ považujeme za nezávislé, a píšeme tedy výsledek měření pro každou vlnovou délku zvlášť $\Delta\beta(\phih, \beta)$ s neznámými aditivními $\xi(\beta)$.

Při změně teploty není nutné můstek znovu vyvažovat, což využíváme k měření teplotních přechodů v oddílu \ref{chap:ferh-field-cooling}.
Pak máme $\Delta\beta(\phih, \beta, T)$ a $\xi(\beta)$.

Základem původní metody zpracování byl vzorec \eqref{eqn:PMLD} platný pro kolmý odraz od izotropního vzorku se saturovanou\footnote{Nepožadujeme \emph{striktní} saturaci, takže $|\vec{M}|=M_S$, ale $\phim\neq\phih$.} in-plane magnetizací popsanou úhlem $\phim$ závisejícím na $\phih$.
V tuto chvíli nepředpokládáme, že by magnetická anizotropie $\phim(\phih)$ byla diktována magnetickou volnou energií Stonerova-Wohlfarthova modelu popsaného v oddílu \ref{chap:magneticka-anizotropie}, naopak předpokládáme libovolnou závislost.
Vzorec \eqref{eqn:PMLD} byl dále naivním způsobem rozšířen pro anizotropní MLD v \cite{wohlrathMagnetooptickaCharakterizaceSpintronickych2018} pomocí $\phim$-závislého koeficientu $P$
\begin{equation}
\label{eqn:PMLD-naivni}
    \Delta\beta(\phih, \beta)=P\left(\phim(\phih)\right) \sin\left(2\phim(\phih)-2\beta\right)+\xi(\beta) \,.
\end{equation}
Zpracování pak spočívalo ve fitování metodou nejmenších čtverců změřeného $\Delta\beta$ neznámými funkcemi\footnote{Vzorkování $\phih$ a $\beta$ je konečné, takže např. pro 72 změřených $\phih^i$ rozumíme jako fitování $\phim(\phih)$ určení 72 parametrů $\phim^i\equiv\phim(\phih^i)$.} $P(\phim)$, $\phim(\phih)$, $\xi(\beta)$.
Metodou nejmenších čtverců rozumíme hledání parametrů minimalizujících cílovou funkci
\begin{equation}
\label{eqn:L}
    \mathcal{L} = \sum_{\phih, \beta} \left( \Delta\beta_\textrm{měření} - \Delta\beta_\textrm{model} \right)^2\,,
\end{equation}
kde v tomto případě $\Delta\beta_\textrm{měření}$, $\Delta\beta_\textrm{model}$, značí levou, resp. pravou stranu \eqref{eqn:PMLD-naivni}.

Problém lze jednoduchou transformací pomocí součtových vzorců převést na lineární\todo{overit}
\begin{align}
\label{eqn:PMLD-linearizovane}
\Delta\beta =\xi(\beta) &+ P\left(\phim(\phih)\right)\sin\left(2\phim(\phih)\right)\cos(2\beta) 
           \\&-P\left(\phim(\phih)\right)\cos\left(2\phim(\phih)\right)\sin(2\beta) \,,
\end{align}
který lze vyřešit rozvedením $\Delta\beta$ a $\xi$ do Fourierových řad\footnote{S konečným vzorkováním se jedná o diskrétní Fourierovu trasformaci, obor $k$ je omezený.} v $\beta$
\begin{align}
    \Delta\beta(\phih, \beta) &= \sum_{k=-\infty}^\infty \Delta\beta_k(\phih) e^{ik\beta} \,,\\
    \xi(\beta) &= \sum_{k=-\infty}^\infty \xi_k e^{ik\beta}  \,.
\end{align}
Řešením je pak
\begin{align}
    \xi_k = \overline{\Delta\beta_{k}} \quad \textrm{pro} \,\, k\neq\pm2 \,,\\
    P(\phim(\phih)) e^{\pm i2\phim(\phih)} + \xi_{\pm2} = \Delta\beta_{\pm2}(\phih) \,,\label{eqn:reseni-ctverce-fourier}
\end{align}
kde $\overline{\Delta\beta_k}$ značí průměr přes $\phih$.
Z \eqref{eqn:reseni-ctverce-fourier} je vidět, že úloha je degenerovaná: existují různé sady parametrů, které produkují stejná naměřená data.
Dimenze degenerace je 2 a odpovídá reálné a imaginární části $\xi_2=\xi_{-2}^*$.

Nicméně pro každou sadu naměřených dat existuje maximálně jedno řešení s čtyřčetnou rotační symetrií v rovině $xy$, tj. $P(\phim)=P(\phim+\SI{90}{\degree})$ a $\phim(\phih+\SI{90}{\degree})=\phim(\phih)+\SI{90}{\degree}$.
Jinými slovy, pro kubické vzorky fit degenerovaný není, platí pro ně totiž vždy $\overline{Pe^{i2\phim}}=0$, takže $Pe^{\pm i2\phim}=\Delta\beta_{\pm2}(\phih)-\overline{\Delta\beta_{\pm2}}$.
Tak lze z měřených dat určit kýžené závislosti $P(\phim)$ a $\phim(\phih)$.

Na úvodních datech CoFe ilustruujeme nedostatky původní verze experimentu.
Měření proběhlo v transmisní geometrii podle schématu na obr. \ref{fig:zakladni-schema} (a) s tím rozdílem, že chyběl ``čistící'' polarizátor P1 a štít kryostatu, takže svazek neprocházel přes skleněná okénka.
Ilustrace změřených dat je na obr. \ref{fig:g-uvod-data} (a).

Z obrázku je na první pohled patrný první problém.
Změřené křivky pro $\beta$ a $\beta+\SI{90}{\degree}$ by měly podle \eqref{eqn:PMLD-naivni} být shodné s opačným znaménkem, což v některých případech zjevně neplatí.
Metoda v takových případech zjevně selhává.

Druhý problém je plíživějšího charakteru.
U některých vlnových délek (především v infračervené oblasti) jsou změřená data dobře popsaná modelem \eqref{eqn:PMLD-naivni} a je možné provést fit \eqref{eqn:reseni-ctverce-fourier}.
Zatímco $P(\phim)$ je v principu spektrálně závislé, od $\phim(\phih)$ očekáváme, že na všech vlnových délkách popisuje tu samou skutečnost.
Výsledky pro některé vlnové délky, na kterých byl fit jinak úspěšný, jsou ale zcela odlišné, viz obr. \ref{fig:g-uvod-data} (b).
V některých případech (např. \SI{405}{\nano\meter}) fit dokonce selhává úplně, protože domnělé $\phim$ se točí na opačnou stranu než $\phih$.

\begin{figure}[htbp]
    \centering
    \includegraphics{./data/out/uvod-data.pdf}
    \caption{Úvodní měření CoFe, $\Hext=\SI{207}{\milli\tesla}$. (a) Měřené stočení na vlnové délce \SI{1050}{\nano\meter} při vybraných polarizacích. Data neodpovídají modelu \eqref{eqn:PMLD-naivni} -- $\beta=\SI{0}{\degree}$ a $\beta=\SI{90}{\degree}$ jsou zcela odlišná, přestože by měla pouze změnit znaménko. (b) Magnetická anizotropie určená různými vlnovými délkami.}
    \label{fig:g-uvod-data}
\end{figure}

Podstata problémů byla odhalena dvojí: ignorace vlivu zrcadel mezi vzorkem a optickým můstkem, a neúplné pochopení anizotropie MLD.\todopn{Je to ok, nebo je to urážlivé?}
Jejich řešení a popisem konečné verze metody se věnuje zbytek této kapitoly.
