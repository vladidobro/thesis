\section{Výchozí situace}
\label{chap:vychozi-situace}

Aparaturou se již zabývali dřívější práce vzniklé na pracovišti.
\todocite{Kimak BP,DP} se věnují charakterizaci magnetu a implementaci měřících procedur.
\todocite{wohlrath bp} uvádí do provozu první verzi optické aparatury a provádí zkušební experimenty, \todocite{kubas bp, kimak Dp} aparaturu vylepšují a využívají k magneto-optickým měřením AFM materiálů.
Charakterizaci některých prvků aparatury se věnovali studentské projekty: \todocite{Badura} měří Faradayův jev v okénkách kryostatu (nežádoucí jev), \todocite{Hovorakova} charakterizuje infračervené \ch{InGaAs} detektory, \todocite{Minar} se zabývá foto-elastickým modulátorem (PEM) a \todocite{Schusser} charakterizuje Berekův kompenzátor charakterizuje Berekův kompenzátor charakterizuje Berekův kompenzátor charakterizuje Berekův kompenzátor.

Žádné z výsledků předchozích prací však nebyly uspokojující.
Úvodním krokem této práce bylo zopakovat a rozšířit spektroskopická měření MLD ve vzorku FR06 ve FM fázi, provedená v \todocite{kubas}.
Měření byla úspěšně replikovaná, ale problémy zůstávali.
