\section{Určení anizotropie MLD}
\label{chap:anizotropie-MLD}

Snaha o vystižení anizotropie MLD pomocí vzorce \eqref{eqn:} založeném na \eqref{} nebyla úspěšná.
Výskyt $\phim$ v argumentu sinu se zakládá na argumentech symetrie; pro plně rotačně symetrické prostředí v rovině prostředí zanáší magnetizace jediný význačný směr a osa zanesené optické anizotropie s ní pak musí splývat.
Tento argument však selhává i pro čtyřčetnou rotační symetrii, pak totiž není důvod aby např. $\phim=\SI{10}{\degree}$ zanášelo optickou anizotropii ve stejném směru.

Správné rozšíření vzorečku by mělo tvar
\begin{equation}
\label{eqn:PMLD-ansatz}
    \Delta\beta = P(\phim) \sin \left( 2\varphi_O(\phim) - 2\beta   \right)
\end{equation}
se směrem optické osy $\varphi_0$, které už pro neúplnou rotační symetrii nemusí rovnat $\phim$.
V takovém tvaru postihuje MO jevy všech řádů (stejně jako původní izotropní vzorec) a trpí stejnou degenerací jako \eqref{eqn:}, takže lze $P$ a $\varphi_O$ určit pouze s nějakým přepokladem (např. čtyřčetné rotační in-plane symetrie), který degeneraci sejme.

Se vzorečkem je ještě jeden problém: $\phim$ v něm vystupuje pouze skrze libovolné funkce $P$ a $\varphi_O$, a bez přidaných požadavků na jejich tvar nelze určit magnetickou anizotropii $\phim(\phih)$.
Shodná magnetická anizotropie je ale ve výsledku jediným spolehlivým kritériem, které nás dokáže přesvědčit o správnosti změřených dat.
Přídavný požadavek na $P$ a $\varphi_O$, který zde přijmeme, je takový, že měřené stočení je MO jevem maximálně kvadratickým v magnetizaci vzorku $\vec{M}$.
Uvidíme, že tento požadavek je k určení magnetické anizotropie dostačující (viz oddíl \ref{chap:urceni-magneticke-anizotropie}), tj. zajišťuje vzájemnou jednoznačnost $\Delta\beta$ a $\phim$.

Vzhledem k libovolnosti funkcí $P$ a $\varphi_O$ je jediná informace obsažená v ansatzu \eqref{eqn:PMLD-ansatz} vzájemná závislost signálu pro různá $\beta$.
Ta je založena na dvou předpokladech.
Zaprvé musí být průchod/odraz přibližně izotropní, tj. nultý člen rozvoje transmisní/reflexní Jonesovy matice (zde obě značíme $\mathcal{R}$) musí být přibližně úměrný jednotkové matici.
V praxi to znamená téměř kolmý dopad a přibližně izotropní tenzor permitivity vzorku.
Označíme odchylku od jednotkové matice $\mathcal{R}'$ a rozdělíme jí na ``nemagnetickou'' a ``magnetickou'' část (po vytknutní celkového faktoru $R_0$).
\begin{align}
\label{eqn:PMLD-Jones}
\mathcal{R}(\vec{M}) &= R_0 \left[ \begin{pmatrix} 1&0\\0&1 \end{pmatrix} + \mathcal{R}'(\vec{M}) \right] \\
                     &= R_0 \left[ \begin{pmatrix} 1&0\\0&1 \end{pmatrix} + \mathcal{R}'_0 + \mathcal{R}'_M(\vec{M}) \right]
\end{align}
Pro jednoduchost zápisu dále vypustíme celkový faktor $R_0$ (pokládáme $R_0=1$), který se ve výpočtu nijak neprojeví.
Pokud je $\mathcal{R}'$ malé, je rozumné uvažovat, že můstek je vyvážen pro stejnou lineární polarizaci, jako je ta dopadající.
Díky tomu je pak možné spočítat, na které prvky reflexní matice je měřený signál citlivý.
Pokud toto splněno není, pak je můstek pro každé $\beta$ citlivý na jiný mix prvků $\mathcal{R}'_M$, který je složitou funkcí nemagnetické $\mathcal{R}'_0$; vzorec pak nemá jednoduchou $\beta$-závislost \eqref{eqn:PMLD-ansatz}.
Po případném elipsometrickém změření $\mathcal{R}'_0$ je možné správnou $\beta$-závislost dopočítat a dosadit do nového ansatzu, my se tím však v této práci nezabýváme, protože uvedený předpoklad splňujeme.

Pomocí Stokesových kovektorů lze jednoduše vyjádřit stočení a elipticitu do prvního řádu v $\mathcal{R}'$.
Pro dané vstupní $\beta$ (Stokesův vektor $\Stks(\beta)$) zavedeme kovektory\footnote{Argument $\beta$ zde značí, pro jaké $\beta$ je daný kovektor platný. Horní index $\beta$ či $\chi$ značí, jestli kovektor měří stočení či elipticitu.} $\Dtks^\beta(\beta)=(0, \cos2\beta, \sin2\beta, 0)$, resp. $\Dtks^\chi(\beta)=(0, 0, 0, 1)$, které měří stočení, resp. elipticitu vyjádřenou změnou Stokesova vektoru $\textrm{d}\Stks$:
\begin{align}
    \Dtks^\beta(\beta) \cdot \Stks(\beta) = 0 \,, \quad &\textrm{d}\beta = \Dtks^\beta \cdot \textrm{d}\Stks \\
    \Dtks^\chi(\beta) \cdot \Stks(\beta) = 0 \,, \quad &\textrm{d}\chi = \Dtks^\chi \cdot \textrm{d}\Stks \,.
\end{align}
Rozvedeme Muellerovu matici \eqref{eqn:} vzorku do prvního řádu\footnote{Striktně vzato po tomto kroku již nejde o mocninný rozvoj měřeného signálu v $\vec{M}$, protože zahazujeme některé kvadratické členy a ponecháváme jiné. Pro malá stočení jsou však zahozené členy silně potlačeny.} v $\mathcal{R}'$ (vyjádřeného diferenciálem $\textrm{d}\mathcal{R}$ v okolí jednotkové matice\footnote{Píšeme $\mathcal{R}= \begin{pmatrix} 1&0\\0&1 \end{pmatrix} + \textrm{d}\mathcal{R}$.}) 
\begin{equation}
\label{eqn:dif-Mueller}
    \textrm{d}M_{ij} = \delta_{ij} \Re\operatorname{Tr}\lbrace\textrm{d}\mathcal{R}\rbrace + \epsilon_{ijk} \Im\operatorname{Tr}\lbrace\sigma_k\textrm{d}\mathcal{R}\rbrace
\end{equation}

Stočení a elipticita jsou
\begin{equation}
    \textrm{d}\beta = \begin{pmatrix} -\sin2\beta&\cos2\beta \end{pmatrix} \begin{pmatrix} \textrm{d}M_{10}&\textrm{d}M_{11}&\textrm{d}M_{12}\\\textrm{d}M_{20}&\textrm{d}M_{21}&\textrm{d}M_{22} \end{pmatrix} \begin{pmatrix} 1\\\cos2\beta\\\sin2\beta \end{pmatrix}
\end{equation}
a
\begin{equation}
    \textrm{d}\chi = \begin{pmatrix} \textrm{d}M_{30}&\textrm{d}M_{31}&\textrm{d}M_{32} \end{pmatrix} \begin{pmatrix} 1\\\cos2\beta\\\sin2\beta \end{pmatrix} \,,
\end{equation}
po dosazení \eqref{eqn:dif-Mueller} pak 
\begin{align}
    \textrm{d}\beta &= -\sin2\beta \ldots + \cos2\beta \ldot + \Im \operatorname{Tr}\lbrace \sigma_3 \textrm{d}\mathcal{R} \,,
    \textrm{d}\chi
\end{align}




Předpokládáme, že vzorek je v jedno-doménovém stavu se saturovanou in-plane magnetizací (Stonerův-Wohlfarthův model z oddílu \ref{chap:magneticka-anizotropie}).
Díky tomu se každý člen řádu $k$ mocninného rozvoje v $\vec{M}$ redukuje na harmonickou funkci $k\phim$.
Do druhého řádu se tedy vyskytují členy $\cos\phim$, $\sin\phim$, $\cos2\phim$, $\sin2\phim$ a konstanta, kterou nedokážeme změřit kvůli vyvažování můstku (konstanta $\xi(\beta)$ z \eqref{eqn:}).
První dva členy jsou lineární a většinou kvůli téměř kolmému dopadu ($< \SI{1}{\degree}$) poměrně malé.

Zajímáme se pouze o kvadratické jevy, proto je od lineárních odseparujeme stejným způsobem jako metoda rotujícího pole (rovnice \eqref{eqn:}).
Předpokládáme in-plane magnetickou anizotropii s dvoučetnou rotační symetrií, pak platí $\phim(\phih+\SI{180}{\degree})=\phim(\phih)+\SI{180}{\degree}$ a symetrizaci podle $\vec{M}$ můžeme provést pomocí symetrizace podle $\vHext$:
\begin{equation}
    \Delta\beta^\textrm{Q}(\phih) = \frac{1}{2}\left(\Delta\beta(\phih) + \Delta\beta(\phih+\SI{180}{\degree})\right)
\end{equation}

V dalším předpokládáme, že symetrizace byla provedena a $\Delta\beta^\textrm{Q}$ značíme bez indexu jako $\Delta\beta$.
Poznamenáme však, že tato symetrizace teoreticky není nezbytná.
V konečném důsledku jsou fitovány koeficienty mocninného rozvoje a principiálně není problém fitovat o několik parametrů navíc.
V této práci jsme se o to nepokoušeli.

Po symetrizaci tedy máme signál, který je lineární
