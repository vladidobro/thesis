\section{FeRh}
\label{chap:vzorek-ferh}

Vzorek FR06 patří do sady vzorků slitiny FeRh připravených naprašováním na \emph{University of California, Berkley} v USA, jejichž studiu se na KChFO MFF UK již věnovali práce \cite{brajerLaserovaSpektroskopieMaterialu2015,saidlUltrarychlaLaserovaSpektroskopie2018,kubascikMagnetooptickeStudiumAntiferomagnetickych2019}.
Konkrétně vzorek FR06 byl podrobně studován v \cite{saidlUltrarychlaLaserovaSpektroskopie2018,kubascikMagnetooptickeStudiumAntiferomagnetickych2019}.
Tento oddíl je čistou kompilací uvedených referencí.

FeRh se vyznačuje magnetickým fázovým přechodem AFM--FM (nižší--vyšší teplota).
Takový fázový přechod sám o sobě není jedinečný, u FeRh k němu však dochází přibližně při \SI{370}{\kelvin}, takže při pokojové teplotě je v AFM fázi, a pro přechod do FM stačí mírný ohřev.
To je velice žádanou vlastností pro spintronické aplikace: v navrhovaných AFM pamětech by informace byla uchovávána při pokojové teplotě v AFM stavu robustním vůči vnějším magnetickým polím, což by mimojiné umožnilo rozsáhlejší miniaturizaci.
Teplotní ``blízkost'' FM fáze pak dovoluje tzv. \emph{heat-assisted magneto-recording} (HAMR): materiál je ohřát, je přiloženo zapisovací magnetické pole, a následně je materiál opět zchlazen.
Tímto způsobem je možné ovlivnit magnetický stav v AFM fázi, a tím do něj zapsat informaci.
Tento jev se označuje jako \emph{field cooling}.
Princip FeRh AFM paměti a field cooling ilustruje obr. \ref{fig:ferh-memory}.
Z ekonomických důvodů však FeRh není perspektivní, protože Rh je příliš málo dostupné.

\begin{figure}[htbp]
    \centering
    \includegraphics{./img/static/ferh-memory-marti.pdf}
    \caption{Ilustrace principu antiferomagnetické FeRh paměti.
        Ve FM stavu při vysoké teplotě (vlevo) magnetizace (černé šipky) následuje přiložené pole $\vec{H}_\textrm{FC}$. 
        Při schlazení dojde k přechodu do různý AFM magnetických uspořádání (barevné šipky). To se projeví změnou odporu. \cite{martiRoomtemperatureAntiferromagneticMemory2014}}
    \label{fig:ferh-memory}
\end{figure}

FeRh má kubickou prostorově centrovanou mřížku.
Struktura a uspořádání mikroskopických momentů je znázorněno na obr. \ref{fig:ferh-struktura}.
Fázový přechod do FM je doprovázen několika projevy: izotropně se zvýší objem krystalu o cca \SI{1}{\percent} a změní se reflektivita.
Při přechodu dochází také ke skokové změně entropie, což vede k teplotní hysterezi, viz obr. \ref{fig:vzorek-ferh} (b).
\todopn{Zvětší nebo zmenší? U nás se zmenšuje, ale Brajer cituje nejaky obrazek kde se zvetsuje.}

\begin{figure}[htbp]
    \centering
    \includegraphics{./img/static/ferh-struktura-saidl.pdf}
    \caption{Struktura FeRh ve AFM (vlevo) a FM (vpravo) fázi.
        Šipky naznačují orientaci magnetických momentů jednotlivých atomů Fe (bílá) a Rh (šedá). \cite{saidlInvestigationMagnetostructuralPhase2016}}
    \label{fig:ferh-struktura}
\end{figure}

\begin{figure}[htbp]
    \centering
    \begin{tikzpicture}
    \begin{scope}[rotate=20,yshift=1cm]
    \filldraw[fill=gray!20] (-4.5,-1) -- (-4.5,1) -- (-1,1) -- (-1,-1) -- cycle;
    \filldraw (-3.3,0) circle [radius=0.2cm];
    \draw (-2.75,0) -- (-2.75,-2);
    \draw[->] (-2.75,0) -- +(-110:2) node[anchor=east] {$x$};
    \draw[->] (-2.75,0) -- +(-20:2.2) node[anchor=south] {$y$};
    \filldraw[fill=green!20,draw=green!50!black] (-2.75,0) -- (-2.75,-1.3cm) arc[start angle=270, end angle=250, radius=1.3cm] -- cycle;
    \path (-2.75,0) ++(-100:1cm) node {$\gamma$};

    \draw[<->] (-4.5,1.2) -- node[anchor=south,rotate=20] {FeRh[110]/MgO[100]} (-1,1.2);

\end{scope}
    \path (3.5,0) node {\includegraphics[width=7cm]{./img/static/ferh-hystereze-kubascik.png}};
    \path (-5,2) node {(a)};
    \path (0,2) node {(b)};
\end{tikzpicture}

    \caption{(a) Vzorek FR06, zavedení úhlu in-plane rotace $\gamma$ a vyznačené předpokládané snadné osy.
    Na vzorku je patrný defekt, který umožňuje snadnou orientaci. (b) Měření odrazivosti při AFM-FM přechodu, měřeno s FR06 v \cite{kubascikMagnetooptickeStudiumAntiferomagnetickych2019}.}
    \label{fig:vzorek-ferh}
\end{figure}

Vzorek FR06 je tvořen \SI{18}{\nano\meter} vrstvou \ch{Fe_{0,5}Rh_{0,5}} na substrátu \ch{MgO}(001) a s krycí vrstvou tantalu\todopn{jak tlustou? Saidl ani Kubaščík to nepíšou, v Brajerovi jsou protiřečící si informace (v textu 1,5 nm, na obrázku 3 nm)}.
Hrana vzorku je shodná s \ch{MgO} [100], \ch{FeRh} [100] je oproti hraně pootočen o \SI{45}{\degree}, viz obr. \ref{fig:vzorek-ferh} (a).

V \cite{brajerLaserovaSpektroskopieMaterialu2015} byla (pro jiný vzorek ze sady -- \SI{36}{\nano\meter} tlustý FR04) údajně měřením MLD v hysterezních smyčkách určena poloha snadných os: \SI{45}{\degree} a \SI{135}{\degree} pootočené od \ch{FeRh} [100] (, které v daném vzorku narozdíl od FR06 údajně splývá s \ch{MgO} [100]).

V \cite{kubascikMagnetooptickeStudiumAntiferomagnetickych2019} bylo ve vzorku FR06 měřeno spektrum MLD v rotujícím poli.
Neuspokojivé výsledky zavdaly vzniku této práce (viz kap. \ref{chap:4}).

