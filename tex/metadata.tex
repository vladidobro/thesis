\thesistype{DIPLOMOVÁ PRÁCE}
\titleCS{Laserová spektroskopie spintronických materiálů}
\titleEN{Laser spectroscopy of spintronic materials}
\yearsubmitted{2022}

\author{Vladislav Wohlrath}
\departmentCS{Katedra chemické fyziky a optiky}
\departmentEN{Department of Chemical Physics and Optics}
\studyprogramme{Fyzika}
\studybranch{Optika a optoelektronika}

\supervisor{prof. RNDr. Petr Němec, Ph.D.}
\supervisordepartmentCS{Katedra chemické fyziky a optiky}
\supervisordepartmentEN{Department of Chemical Physics and Optics}

\dedication{Tímto děkuji vedoucímu prof. RNDr. Petrovi Němcovi, Ph.D. za cenné zkušenosti, poskytnuté příležitosti a obecně férový přístup.
Také za dohled nad prováděným experimentem a zkušené rady při tvorbě textu.

Děkuji doc. RNDr. Tomášovi Ostatnickému, Ph.D. za teoretickou podporu a mnohé plodné konzultace.

Děkuji Zeynab Sadeghiové, MSc. za spolupráci při měření závěrečné sady dat se vzorkem FeRh.

Děkuji RNDr. Lukášovi Nádvorníkovi, Ph.D. za poskytnutí měřeného vzorku CoFe.

Děkuji Mgr. Jozefovi Kimákovi za uvedení do probíhajícího experimentu a také za pochopení a vstřícný přístup při mém opakovaném zapomínání klíčů od laboratoře, kterým jsem ho obtěžoval.

Děkuji kolektivu pracovníků a studentů působících na pracovišti, kromě již zmíněných jmenovitě RNDr. Evě Schmoranzerové, Ph.D., Mgr. Peterovi Kubaščíkovi, Mgr. Miloslavovi Surýnkovi, Bc. Jiřímu Jechumtálovi a Deepovi Joshimu, MSc.
Děkuji jim za inspiraci, občasnou pomoc při práci a hlavně za zpříjemnění stráveného času.

Děkuji rodině a přátelům za podporu.
Zvláštní dík si zaslouží Bc. Jaroslav Pešek, v jehož inkubátoru diplomových prací vznikla podstatná část tohoto textu.}

\abstractCS{Na pracovišti je dlouhodobě vyvíjena metoda studia magneto-optických jevů kvadratických v magnetizaci vzorku (Voigtův jev). 
Kvadratická magneto-optika je jednou z mála dostupných metod studia antiferomagnetů, a proto je důležitým nástrojem spintronického výzkumu.
Oproti podobným metodám je v našem případě díky rozdílné geometrii experimentu umožněno použití kryostatu.
V rámci této práce bylo identifikováno a odstraněno několik problémů, které použití doposud znemožňovaly.
Použití metody bylo demonstrováno v transmisní i reflexní geometrii se vzorky CoFe a FeRh.}
\abstractEN{A method for the study of magneto-optic effects quadratic in sample magnetization (Voigt effect) is being developed at our department.
Quadratic magneto-optics is one of not many available methods suitable for the study of antiferromagnets, which makes it an important tool for spintronics research.
Compared to other similar methods, we use a different geometry which allows for the use of cryostat.
In this thesis, we identified and ultimately solved several problems that prevented the use of the method in the past.
Then, we successfully demonstrated the method both in transmission and reflection geometry with CoFe and FeRh samples.}
\keywordsCS{spintronika, kvadratická magneto-optika, Voigtův jev, CoFe, FeRh}
\keywordsEN{spintronics, quadratic magneto-optics, Voigt effect, CoFe, FeRh}
