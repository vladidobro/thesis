\documentclass{beamer}

\title{Framework pro automatické zlepšování klasifikace síťového provozu}
\author{Bc. Jaroslav Pešek}
\institute{Fakulta informačních technologií ČVUT \\ \vspace{0.2cm} Vedoucí: Ing. Dominik Soukup}
\date{8. června 2022}

\usetheme{default}
\usecolortheme{CVUTFIT}
\beamertemplatenavigationsymbolsempty

\setbeamertemplate{footline}[frame number]


\begin{document}

\frame[plain]{\titlepage}

\begin{frame}{Motivace}

\begin{itemize}
    \item V rámci bezpečnostní agendy v síti, klasifikace zapadá do části sběru informací a hlášení potenciálních anomálií
    \item Některé protokoly mohou být zapouzdřeny v jiném (příklad: DNS over HTTPS), deterministická (klasické DNS) klasifikace je složitá

    \item Využití ML při klasifikaci síťového provozu je dobře popsáno, ale:
    \begin{itemize}
        \item velmi závisí na datové sadě --- kde ji vzít?
        \item sada rychle zastarává --- jak ji dynamicky aktualizovat?
        \item komunikace je šifrovaná --- jak tuto vlastnost obejít?
    \end{itemize}
\end{itemize}

\end{frame}


\begin{frame}{Základní nástroje}
    
\begin{itemize}
    \item IP Flows
    \begin{itemize}
        \item agregované informace z množiny paketů jednoho spojení v nějakém časovém rozmezí $t_1$ až $t_2$
        \item nezájem o vlastní data --- invariantní vůči šifrování
    \end{itemize}
    \item Machine learning
    \begin{itemize}
        \item Oblast na pomezí statistiky a umělé inteligence zobecňující charakteristiky vstupních dat  
    \end{itemize}
    \item Active learning
\end{itemize}
\end{frame}

\begin{frame}{Active learning}
    \begin{center}
    \includegraphics[width=0.4\textwidth]{../text/analyza/stream.pdf}
    \end{center}
    \begin{itemize}
        \item O anotaci a přidání toku do datové sady rozhoduje vzorkovací strategie
        \begin{itemize}
            \item založena na náhodném výběru, nejistotě modelu nebo vzdálenosti
        \end{itemize}
        \item \textbf{Ne}anotované vzorky zahazujeme
        \item Přirozeně \textit{stream-oriented} a emuluje \textit{online} ML
    \end{itemize}    
\end{frame}


\begin{frame}{Architektura řešení}
\begin{center}
\includegraphics[width=0.5\textwidth]{../text/navrh/activity.pdf}
\end{center}

\end{frame}

\begin{frame}{Řešení}
    \begin{itemize}
        \item Architektura je vysoce modulární a jednotlivé součásti mají rigidně definované API
        \item Navržena nezávisle na zdroji
        \item Implementováno s ohledem na definované funkční a nefunkční požadavky vyplývající z analýzy problému
        \item Zásadní část je vzorkovací strategie --- byla implementována řada strategií, v rámci frameworku proběhlo jejich experimentální vyhodnocení
    \end{itemize}
\end{frame}

\begin{frame}{Ukázka experimentu}
\begin{center}
\includegraphics[width=0.7\textwidth]{../text/test/doh_online_2.pdf}
\end{center}



\end{frame}


\begin{frame}{Shrnutí}
\begin{itemize}
        \item Byly experimentálně vyhodnoceny vhodné strategie pro proudový active learning, některé zjištěné poznatky jsou pro další výzkum cenné:
        \begin{itemize}
            \item řada slibných vzorkovacích strategií je nevhodná pro proudové zpracování --- nerozumně vysoká výpočetní složitost
            \item náhodná strategie je v souladu s poznatky velmi výkonná, je však snadno překonána --- další výzkum strategií má smysl
            \item vzorkování založeno na podobnosti toků je pomalé a spíše méně výkonné, naopak ty založeny na nejistotě modelu jsou velmi rychlé a výkonné -- dobrá zpráva
        \end{itemize}
    \item Právě teď: nasazeno v síti CESNET jako modul NEMEA
    \item Stále aktivní vývoj, vznikají další části -- zpětná vazba lidským expertem, postprocessing, kontrola kvality trénovacích sad
    \item Součást programu Strategická podpora rozvoje bezpečnostního výzkumu ČR (MVČR), projekt FETA    
    \item Odeslaný článek na PESW (Prague Embedded Systems Workshop) a finalizace článku na CNSM (International Conference on Network and Service Management)


\end{itemize}
\end{frame}

\begin{frame}{Otázky}
\begin{block}{Otázka oponentky 1}
V práci zmiňujete kratší výpadky v průběhu testování, způsobeny neodladěnými chybami - jaké to byly chyby, a jak jste se s nimi vypořádal?
\end{block}

\begin{exampleblock}{Odpověď}
Týká se prvního online experimentu, který proběhl ve fázi raného vývoje. Havárie v průběhu referovaného experimentu byly dvě:
\begin{enumerate}
    \item Špatně ošetřený vstup vedl k neodchycené výjimce, která způsobila pád programu. Oprava triviální.
    \item Chyba v implementaci balíčku \texttt{pytrap} z projektu NEMEA. Tato chyba se nachází mimo ALF, je v řešení. Oprava je v podobě nastavení zotavení na úrovni OS. Toto řešení je korektní, protože ALF si v chodu vytváří auxiliární soubory a nedochází tak ke ztrátě dat.
\end{enumerate}
\end{exampleblock}
\end{frame}

\begin{frame}{Otázky}
\begin{block}{Otázka oponentky 2}
V čem spatřujete hlavní přínosy Active Learning pro detekci bezpečnostních hrozeb v síťovém provozu? Má tato metoda i potenciální slabiny?
\end{block}

\begin{exampleblock}{Odpověď}
\begin{description}
    \item[přínosy] aktualizace datových sad, využití ML metod pružněji, lze monitorovat anomálie a reportovat je
    \item[slabiny] kopírují ML obecně -- významné riziko chyb 1. i 2. druhu, riziko deformace trénovací sady vlivem zkreslení modelu -- riziko spirály vhledem k inkrementační povaze metody
\end{description}
\end{exampleblock}
\end{frame}

\begin{frame}{}
\centering \huge Diskuze 
\end{frame}









\end{document}

