\chapter*{Závěr}
\addcontentsline{toc}{chapter}{Závěr}

V práci jsme se věnovali studiu magnetooptických jevů v dobře prostudovaném vzorku feromagnetického polovodiče GaMnAs pomocí nově postaveného prototypu 2D elektromagnetu.

Nejprve jsme ověřili použitelnost elektromagnetu přesným zopakováním experimentů provedených v práci \cite{Reichlova} s jiným elektromagnetem. Jednalo se o měření Voigtova jevu a MLD v hysterezních smyčkách. Poté jsme vyzkoušeli novou variantu experimentálního uspořádání, které nám umožňuje kolmý dopad světla na vzorek. Nakonec jsme vyzkoušeli novou metodu, ve které při konstantní velikosti vnějšího pole měníme jeho směr.
Naše měření jasně ukázala, že pomocí Voigtova jevu a MLD je možné velice efektivně určit magnetickou anizotropii vzorku (konkrétně určit polohu snadných os magnetizace). Měření využívající změnu směru konstantního magnetického pole dále umožňuje od sebe oddělit na krystalografickém směru závislé a nezávislé složky příslušného magnetooptického koeficientu.
