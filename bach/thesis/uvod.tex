\chapter*{Úvod}
\addcontentsline{toc}{chapter}{Úvod}

Spintronika je obor elektroniky, který pro přenos a uchování informace používá kromě náboje také spin elektronů.

V roce 2017 byl v Laboratoři OptoSpintroniky na MFF UK uveden do provozu prototyp dvoudimenzionálního elektromagnetu, který umožňuje při stálé velikosti pole měnit jeho směr. Tato práce přímo navazuje na bakalářskou práci Josefa Kimáka (\cite{Kimak}), který provedl jeho charakterizaci.

Hlavním cílem práce bylo postavit experimentální uspořádání pro magnetooptickou charakterizaci magnetických materiálů pomocí dvoudimenzionálního magnetu při různých teplotách. Dále provést úvodní magnetooptická měření na známém vzorku GaMnAs pro ověření použitelnosti magnetu.

Posledním cílem práce je vyzkoušet nové experimentální metody pro měření Voigtova jevu a magnetického lineárního dichroismu (MLD) v reflexní geometrii. Konkrétně se jedná o geometrii kolmého dopadu světla na vzorek (do současnosti se používal pouze téměř kolmý dopad) a plné využití dvoudimenzionality elektromagnetu.

Tyto experimentální metody budou v budoucnu využity pro studium materiálů vhodných pro spintroniku.