\documentclass[12pt]{report}

\usepackage[a4paper, hmargin=1in, vmargin=1in]{geometry}
\usepackage[a-2u]{pdfx}
\usepackage[czech]{babel}
\usepackage{luavlna}

\begin{document}

For a relatively long time, a new experimental technique for the study of magnetic materials by magneto-optical effects quadratic in sample magnetization, like Voigt effect, is being developed at Laboratory of OptoSpintronics. Thanks to the used experimental geometry, in our setup, unlike in different methods, it is possible to use a cryostat. Thanks to this the investigated samples can be studied both at low and high temperatures. In this master thesis we identified and eventually solved several problems that prevented the practical utilization of this technique in the past. We successfully demonstrated the application of this method both in transmission and reflection geometries using ferromagnetic samples of CoFe and FeRh. Our measurements revealed that the coefficient describing the quadratic magneto-optical response can be strongly anisotropic with a wavelength-dependent magnitude and sign. This observation has strong consequences for the design and/or interpretation of experiments based on quadratic magneto-optics, which are gaining an increasing attention nowadays thanks to the growing interest in antiferromagnetic spintronics.

\end{document}
